\documentclass[notitlepage]{article} 
\usepackage[MeX]{polski}
\usepackage[utf8]{inputenc} \usepackage{anysize}
\usepackage{graphicx}
\usepackage{enumerate} 
\usepackage{amsmath} 
\usepackage{amssymb} 
\usepackage{amsfonts}
\usepackage{amsthm}
\marginsize{3cm}{2cm}{1cm}{1cm}

\newtheorem{theorem}{Theorem}[section]
\newtheorem{lemma}[theorem]{Lemma}
\newtheorem{proposition}[theorem]{Proposition}
\newtheorem{corollary}[theorem]{Corollary}


\theoremstyle{definition}
\newtheorem*{definition}{Definition}
\newtheorem{example}{Example}
\newtheorem{remark}[theorem]{Remark}

\newcommand\Spec{\operatorname{Spec}}
\newcommand\Proj{\operatorname{Proj}}
\newcommand\Pic{\operatorname{Pic}}
\newcommand\rad{\operatorname{rad}}
\newcommand\rank{\operatorname{rank}}
\newcommand\im{\operatorname{im}}
\newcommand\id{\operatorname{id}}
\newcommand\sheafhom{\mathcal{H}om}
\newcommand\coker{\operatorname{coker}}
\newcommand\codim{\operatorname{codim}}
\newcommand\Hom{\operatorname{Hom}}
\newcommand\Ext{\operatorname{Ext}}
\newcommand\sspan{\operatorname{span}}
\newcommand\chark{\operatorname{char}}
\renewcommand\O{\mathcal{O}}
\renewcommand\P{\mathbb{P}}
\newcommand\A{\mathbb{A}}
\newcommand\C{\mathbb{C}}
\newcommand\I{\mathcal{I}}
\newcommand\Q{\mathbb{Q}}
\newcommand\Z{\mathbb{Z}}
\newcommand\N{\mathbb{N}}
\newcommand\E{\mathcal{E}}
\newcommand\F{\mathcal{F}}
\newcommand\GG{\mathbb{G}}
\newcommand\LL{\mathcal{L}}
\newcommand\p{\mathcal{P}}

\title{Solutions to Ireland, Rosen ``A Classical Introduction to Modern Number Theory''}
\author{Adam Michalik} 

\begin{document}
\maketitle
\section{Chapter 1}

\paragraph{Ex. 1.1}

Let $a$ and $b$ be nonzero integers. We can find nonzero integers $q$
and $r$ such that $a = qb + r$ where $0 \leq r < b$. Prove that $(a, b) = (b, r)$

As a reminder, $(a_1, \ldots, a_n)$ is defined to be the ideal
generated by $a_i$, but also sometimes by abuse of notation it denotes
the smallest positve member of the ideal (which generates it).

The relation $a = qb + r$ shows that $a \in (b, r)$, so $(a, b) \subset (b, r)$.
On the other hand, $r = qb - a$, so $r \in (a, b)$, whus $(r, b) \subset (a, b)$.

\paragraph{Ex. 1.2}

XXX Exercise statement missing.

The only thing here that needs proving is that the process finishes in
finitely many steps, but this is clear, as $r_i > r_{i+1} \geq 0$ by
construction, so $r_i$ is a decreasing sequence of real numbers, which
cannot be infinite.

\paragraph{Ex. 1.3}

Calculate $(187, 221)$, $(6188, 4709)$, $(314, 159)$.

Use the method from Ex. 1.2. Calculation omitted.

\paragraph{Ex. 1.4}

Let $d = (a, b)$. Show how one can use the Euclidean algorithm to find
numbers $m$ and $n$ such that $am + bn = d$.

The method from Ex. 1.2 produces a sequence $r_i$, such that $r_{k+1}
= d$, and $r_{k+2} = 0$, that is, $r_{k+1} | r_k$. We have
\begin{equation}
  \label{eq:eucl-1}
  r_{k-1} = q_{k+1} r_{k} + r_{k+1},
\end{equation}
so
\begin{equation}
  \label{eq:eucl-2}
  r_{k+1} = r_{k-1} - q_{k+1} r_{k}.
\end{equation}
Next, we have
\begin{equation}
  \label{eq:eucl-3}
  r_{k-2} = q_{k} r_{k-1} + r_{k},
\end{equation}
so
\begin{equation}
  \label{eq:eucl-4}
  r_{k} = r_{k-2} - q_{k} r_{k-1}.
\end{equation}
Substituting (\ref{eq:eucl-4}) back to (\ref{eq:eucl-2}) allows us to
express $d = r_{k+1}$ in terms of $r_{k-1}, r_{k-2}$. Continuing this
procedure will allow us to express $d$ in terms of $r_i, r_{i-1}$, and
finally in terms of $r_1, r_0$, which can be expressed in terms of $a$
and $b$.

\paragraph{Ex. 1.5}

Find $m$ and $n$ for the pairs $a$ and $b$ given in Ex 1.3

Skipped.

\paragraph{Ex. 1.6}

Let $a, b, c \in \Z$. Show that the equation
\begin{equation}
  \label{eq:linear-dioph}
  ax + by = c
\end{equation}
has solutions in integers iff $(a, b) |c$.

From $ax + by = c$ it instantly follows that any common divisor of $a$
and $b$ also divides $c$, so $(a, b) | c$. On the other hand, if $(a,
b)|c$, let $d = (a, b)$, and write $c = dd'$ for $d' \in
\Z$. Write $am + bn = d$ for $m, n \in
\Z$. Multiplying by $d'$ gives us $amd' + bnd' = dd' = c$. We
see that this gives a solution to (\ref{eq:linear-dioph}) -- set $x =
md', y = nd'$.

\paragraph{Ex. 1.7}

Let $d = (a, b)$ and $a = da'$ and $b = db'$. Show that $(a', b') = 1$.

Let $d' | a'$. Since $a = da'$, this means that $d'd | a$. Similarly,
if $d' | b'$, we show that that $d' d | b$. This gives us $d' = 1$ --
otherwise, $d' d$ would have been a greater common divisor of $a, b$.

\paragraph{Ex. 1.8}

Let $x_0$ and $y_0$ be a solution to $ax + by = c$. Show that all
solutions have the form $x = x_0 + t(b/d)$, $y = y_0 - t(a/d)$, where $d =
(a, b)$ and $t \in \Z$.

We have $a(x-x_0) + b(y-y_0) = 0$. Clearly, $a/d | a(x-x_0)$, so also
$a/d | b(y-y_0)$. Since $(b/d, a/d) = 1$ by previous exercise, we also
$(b, a/d) = 1$, therefore $a/d$ must divide $y-y_0$. Similarly, $b/d$
must divide $x-x_0$. Let $x-x_0 = t(b/d), y-y_0 = t'(a/d)$. We have
$at(b/d) + bt'(a/d) = 0$, so $t' = -t$.

\paragraph{Ex. 1.9}
Suppose that $u, v \in \Z$ and that $(u, v) = 1$. If $u | n$
and $v | n$, show that $uv | n$. Show that this is false if $(u, v) \ne 1$.

If $(u, v) = d \ne 1$, let $n = u (v/d) = (u/d) v$. Clearly $u | n$
and $v | n$. On the other hand, $uv > n$, so it cannot divide $n$.

Now let $(u, v) = 1$, and $u|n$, $v|n$. Write $n = au = bv$. Since
$v|n$, $v | au$. Since $(u, v) = 1$, by proposition 1.1.1, $v | a$,
that is, $a = a'v$. Thus, $n = a'vu$, so $vu | n$.

\paragraph{Ex. 1.10}
Suppose that $(u, v) = 1$. Show that $(u+v, u-v)$ is either 1 or 2.

Let $d = (u+v, u-v)$. Since $d | u+v$ and $d | u-v$, by taking sum and
difference, $d | 2u$ and $d | 2v$. Take any prime $p | d$, it also
divides both $2u$ and $2v$. If $p > 2$, by proposition 1.1.1, $p|u$
and $p|v$, which contradicts $(u, v) = 1$.

\paragraph{Ex. 1.11}
Show that $(a, a+k)|k$.

Let $d |a$, $d | a+k$. It also must divide their difference, that is, $a+k - a = k$.

\paragraph{Ex. 1.12}

Suppose that we take several copies of a regular polygon and try to fit them evenly
about a common vertex. Prove that the only possibilities are six equilateral triangles,
four squares, and three hexagons.

Consider an arrangment of $k$ $n$-gons evenly about a common
vertex. The sum internal angle in a regular n-gon is $\pi - 2\pi/n =
(n-2)\pi/n$. Thefore, we must have $2\pi = k(n-2)\pi/n$, so $2n =
k(n-2)$, thus $2n + 2k = kn$. By symmetry, we can assume that $n \leq
k$. Then, $kn = 2k + 2n \leq 2k + 2k = 4k$, so $n \leq 4$, and thus $n
= 3$ and $k = 6$, or $n = 4$ and $k = 4$. From the symmetrical case,
we get $k = 3$, $n = 6$.

\paragraph{Ex. 1.13}
Skipped.

\paragraph{Ex. 1.14}
Skipped.

\paragraph{Ex. 1.15}
Prove that $a \in \Z$ is the square of another integer iff
$ord_p(a)$ is even for all primes $p$.  Give a generalization.

A generalization is of course ``$a$ is $n$-th power if $ord_p(a)$ is
divisible by $n$ for all primes $p$. There's a slight error in the
question -- one also needs to look at the sign of $a$ -- $-4$ is not a
square of another integer, even though $ord_p(a)$ is even for all
primes $p$. We'll therefore assume that $a \in \N$.

The exercise is obvious once we look at the unique factorization -- we have:
\begin{equation}
  a = \prod_{p|a,\, p\,\textrm{prime}} p^{ord_p(a)}
\end{equation}
Since $ord_p(a)$ are even, $ord_p(a)/2$ are integers, so:
\begin{equation}
  a = \prod_{p|a,\, p\,\textrm{prime}} (p^{ord_p(a)/2})^2 = \left(\prod_{p|a,\, p\,\textrm{prime}} p^{ord_p(a)/2}\right)^2
\end{equation}
For the other direction, if $a = b^2$, then clearly $ord_p(a) =
2ord_p(b)$. The proof generalizes for $n > 2$.

\paragraph{Ex. 1.16}
If $(u, v) = 1$ and $uv = a^2$ , show that both $u$ and $v$ are squares.

By previous exercise, and by symmetry, it's enough to prove that
$ord_p(u)$ is even. Now, $ord_p(a^2) = ord_p(u) + ord_p(v)$, so $2
ord_p(a) = ord_p(u) + ord_p(v)$. One of $ord_p(u), ord_p(v)$ must be
$0$, otherwise $p$ divides both $u$ and $v$, which contradicts $(u, v)
= 1$. Thus, either $ord_p(u) = 0$, which is even, or $ord_p(u) = 2
ord_p(a)$, which is even too.

\paragraph{Ex. 1.17}
Prove that the square root of 2 is irrational, i.e., that there is no rational number
$r = a/b$ such that $r^2 = 2$.

Follows from Ex 1.18

\paragraph{Ex. 1.18}
Prove that $\sqrt[n]{m}$ is irrational if $m$ is not the $n$-th power of an integer.

Let $r = a/b$ be such that $r^n = m$. Assume $r$ is in lowest terms,
that is, $(a, b) = 1$. It necessarily follows that $(a^n, b^n) = 1$,
so $m = r^n = a^n/b^n$ is also in lowest terms. Thus, if $m$ is an
integer, $b^n = 1$, so $b = 1$, and $r$ is also an integer.

\paragraph{Ex. 1.19}
Define the least common multiple of two integers $a$ and $b$ to be an integer $m$ such that
$a|m$, $b|m$, and $m$ divides every common multiple of $a$ and $b$. Show that such an $m$
exists. It is determined up to sign. We shall denote it by $[a, b]$.

Consider a set $I = \{n \in \Z: a|n \land b|n\}$. Clearly it
is an ideal of $\Z$, and so $I = (m)$ for some $m$.

\paragraph{Ex. 1.20} Skipped.
\paragraph{Ex. 1.21} Skipped.
\paragraph{Ex. 1.22} Skipped.

\paragraph{Ex. 1.23} 
Suppose that $a^2 + b^2 = c^2$ with $a, b, c \in \Z$ For example, $3^2 + 4^2 = 5^2$ and $5^2 +
12^2 = 13^2$ . Assume that $(a, b) = (b, c) = (c, a) = 1$. Prove that there exist integers $u$
and $v$ such that $c - b = 2u^2$ and $c + b$ = $2v^2$ and $(u, v) = 1$ (there is no loss in
generality in assuming that $b$ and $c$ are odd and that $a$ is even). Consequently $a = 2uv$,
$b = v^2 - u^2$ , and $c = v^2 + u^2$ . Conversely show that if $u$ and $v$ are given, then the
three numbers $a$, $b$, and $c$ given by these formulas satisfy $a^2 + b^2 = c^2$.

If $a^2 + b^2 = c^2$, we have $a^2 = c^2 - b^2 = (c-b)(c+b)$. By Ex
1.10, $(c-b, c+b)$ is either $1$ or $2$. Since both $b, c$ are odd,
both $c-b, c+b$ are even, and so $(c-b, c+b) = 2$. Thus, $c-b = 2x$,
$c+b = 2y$, and $(x, y) = 1$. Consider any prime $p \ne 2$. As $a^2 =
c^2 - b^2 = (c-b)(c+b)$, and $ord_p(c-b) = ord_p(x), ord_p(c+b) =
ord_p(y)$ for $p \ne 2$, it follows that $2 ord_p(a) = ord_p(c-b) +
ord_p(c+b) = ord_p(x) + ord_p(y)$, and one of the $ord_p(x), ord_p(y)$
must be $0$, as $(x, y) = 1$. Thus $ord_p(x)$ and $ord_p(y)$ are even
for all $p$, and so $x = u^2$, $y = v^2$.

The other direction is simple calculation.

\paragraph{Ex. 1.24} Skipped.

\paragraph{Ex. 1.25}
If $a^n - 1$ is a prime, show that $a = 2$ and that $n$ is a
prime. Primes of the form $2^p - 1$ are called Mersenne primes. For
example, $2^3 - 1 = 7$ and $2^5 - 1 = 31$. It is not known if there are
infinitely many Mersenne primes.

Let $a^n - 1$ be prime. We have:

\begin{equation}
  a^n - 1 = (a-1)(a^{n-1} + a^{n-2} + \ldots + a + 1)
\end{equation}

It follows that $a-1|a^n - 1$, but as $a^n - 1$ is prime, $a-1 = 1$,
and so $a = 2$.

Similarly, let $n = uv$, and let $u \leq v$. We have:
\begin{equation}
  2^n - 1 = 2^{uv} - 1 = (2^{u})^v - 1 = (2^u-1)((2^u)^{v-1} + (2^u)^{v-2} + \ldots + 2^u + 1)
\end{equation}
We thus have that $2^u - 1 | 2^n - 1$, but since $2^n - 1$ is prime,
$2^u -1 = 1$, so $u = 1$, and therfore $n$ is prime.

\paragraph{Ex. 1.26}
If $a^n + 1$ is a prime, show that $a$ is even and that $n$ is a power
of 2. Primes of the form $2^{2^t} + 1$ are called Fermat primes. For
example, $2^{2^1} + 1 = 5$ and $2^{2^2} + 1 = 17$.  It is not known if
there are infinitely many Fermat primes.

Let $a^n + 1$ be prime. If $a$ is odd, $a^n + 1$ is even, so it is
prime only if $a = n = 1$ (this is a minor mistake in the statement of
the problem). Assume now that $a$ is even. Suppose that $p | n$, $p$
is odd. Then, letting $n = pv$ for some $v$, we have:

\begin{equation}
  a^n + 1 = a^{pv} + 1 = (a^v)^p + 1 = (a^v+1)((a^u)^{p-1} + (a^v)^{p-2} + \ldots + a^v + 1)
\end{equation}

So, $a^v + 1|a^n + 1$, and therefore $a^n + 1$ is not prime.

\paragraph{Ex. 1.27}
For all odd $n$ show that $8| n^2 - 1$. If $3$ doesn't divide $n$,
show that $6| n^2 - 1$.

We have $n^2 - 1 = (n+1)(n-1)$. Since $n$ is odd, both $n+1, n-1$ are
even, and moreso, one of these must be divisible by 4, as one of the
two consecutive odd numbers is divisible by 4. Thus, their product is
divisible by $8$. Similarly, if $3$ does not divide $n$, it must
divide one of $n-1, n+1$, otherwise it wouldn't divide three
consecutive integers, which is impossible. As $n$ is odd, $n+1$ is
even, so $(n+1)(n-1)$ is divisible by both $2$ and $3$, so it is
divisible by $6$.

\paragraph{Ex. 1.28}
For all $n$ show that $30 | n^5 - n$ and that $42| n^7 - n$.

The reasoning here is very similar to the previous exercise -- for the
first, we use $30 = 2\cdot3\cdot5$ with $n^5 - n = n(n^4 - 1) =
n(n^2-1)(n^2+1) = n(n-1)(n+1)(n^2 + 1)$. It necessarily follows that
one of $n-1, n, n+1$ is divisible by 2, one by 3, and if none of them
is divisible by 5, it follows that $n$ gives the remainder of 2 or 3
when divided by 5. But then, $n^2$ gives the rest of $2^2 = 4$ in the
first case, and $3^2 \mod 5 = 4$ in the second case, so in both cases
$n^2 + 1$ is divisible by 5. We omit the similar argument for $42| n^7
- n = n(n^6-1) = n(n^2 - 1)(n^2 + n + 1)$.

\paragraph{Ex. 1.29}
Suppose that $a, b, c, d \in \Z$ and that $(a, b) = (c, d) =
1$. If $(a/b) + (c/d) =$ an integer, show that $b = d$ or $b = -d$.

We have:

\begin{equation}
  \frac{a}{b} + \frac{c}{d} = \frac{ad + bc}{bd}
\end{equation}

Assume that it is an integer, that is, $bd | ad + bc$. Since $d|bd$
and $d|ad$, we also have $d|bc$, but $(c, d) = 1$, so $d|b$. Similarly
we argue that $b|d$. The thesis follows.

\paragraph{Ex. 1.30}
Prove that

\begin{equation}
  H_n = 1+ \frac{1}{2} + \frac{1}{3} + \ldots + \frac{1}{n}
\end{equation}

is not an integer.

Let $2^s$ be the largest power of $2$ occuring as a denominator in
$H_n$, say $2^s = k \leq n$. Write $H_n = \frac{1}{2^s} + (1 + 1/2 +
\ldots + 1/(k-1) + 1/(k+1) + \ldots + 1/n$. The sum in parentheses can
be written as $1/2^{s-1}$ times sum of fractions with odd
denominators, so the denominator of the sum in parentheses will not be
divisible by $2^s$, but it must equal $2^s$ by Ex 1.29.

\paragraph{Ex. 1.31}
Show that $2$ is divisible by $(1+i)^2$ in $\Z[i]$.

We have $(1+i)^2 = 1 + 2i -1 = 2i$, so $2 = -i(1+i)^2$.

\paragraph{Ex. 1.32}
For $\alpha = a + bi \in \Z[i]$ we defined $\lambda(\alpha) =
a^2 + b^2$. From the properties of $\lambda$ deduce the identity $(a^2
+ b^2)(c^2 + d^2) = (ac - bd)^2 + (ad + bc)^2$.

It is not clear what exactly properties of $\lambda$ they mean, but
most likely the fact that $\lambda(ab) = \lambda(a)\lambda(b)$, which
is never stated in the text, but amounts to proving this very
identity.

\paragraph{Ex. 1.33}
Show that $\alpha \in \Z[i]$ is a unit iff  $\lambda(\alpha) = 1$. Deduce that 1, -1, i, and - i are the only units in $\Z[i]$.

If $\lambda{\alpha} = 1$, then we must have $a^2 + b^2 = 1$, and so
either $a^2 = 1, b^2 = 0$, in which case either $a = 1$ or $a = -1$,
or $a^2 = 0, b^2 = 1$, in which case $b = 1$ or $b = -1$. These 4
options give us $1, -1, i, -i$, all of which are units of $\Z[i]$.

In the other direction, let $\alpha$ be a unit, that is, there exists
$\beta$ such that $\alpha \beta = 1$. We have $1 = \lambda(1) =
\lambda(\alpha\beta) = \lambda(\alpha)\lambda(\beta)$, and so
$\lambda(\alpha) = 1$.

\paragraph{Ex. 1.34}
Show that 3 is divisible by $(1 - \omega)^2$ in $\Z[\omega]$.

We have $(1 - \omega)^2 = 1 - 2\omega + \omega^2$. Now, $\omega =
(-1+\sqrt{-3})/2$ and $\bar{\omega} = \omega^2$, so $1 - 2\omega +
\omega^2 = 2 - \sqrt{-3}+(-1-\sqrt{-3})/2 = 3(1-\sqrt{3})/2 =
3\cdot(-\omega)$. Now, $\omega$ is a unit of $\Z[\omega]$, so
$3 = -(1 - \omega)^2 \cdot \omega^{-1}$

\paragraph{Ex. 1.34}
For $\alpha = a + b\omega \in \Z[\omega]$ we defined
$\lambda(\alpha) = a^2 - ab + b^2$. Show that $\alpha$ is a unit iff
$\lambda(\alpha) = 1$. Deduce that $1, -1, \omega, -\omega, \omega^2 ,
and -\omega^2$ are the only units in $\Z[\omega]$.

It is enough to show that $\lambda$ is multiplicative. We have:

\begin{equation}
  \alpha = a + b\omega = (2a-b)/2 + ib\sqrt{3}/2
\end{equation}

Thus $|\alpha|^2 = (4a^2 - 4ab +b^2)/4 + 3b^2/4 = a^2 - ab + b^2$, so
$\lambda$ coincides with square of complex absolute value, which is
multiplicative.

\paragraph{Ex. 1.39}
Show that in any integral domain a prime element is irreducible.

Let $p$ be a prime element of $A$, that is, $(p) \subset A$ is a prime
ideal. Suppose $p = ab$. Since $ab \in (p)$, and $(p)$ is prime,
either $a$ or $b$ is in $(p)$, say $a$. Then $a = px$ for some $x \in
A$. We then have $p = ab = pxb$, so $p(1-bx) = 0$. Since $A$ is
integral domain, $1-bx = 0$, so $1 = bx$, that is, $b$ is a unit, and
therefore $p$ is irreducible.

\section{Chapter 2}

\paragraph{Ex. 2.1}
Show that $k[x]$, with $k$ a finite field, has infinitely many
irreducible polynomials.

Let $f_1, \ldots, f_n$ be a finite set of polynomials in
$k[x]$. Consider $f = 1+ \prod f_i$. It is not divisible by any of
$f_i$, so none of its irreducible factors can be equal to any of the
$f_i$. Therefore $f_1, \ldots, f_n$ is not the list of all irreducible
polynomials in $k[x]$.

\paragraph{Ex. 2.2}

The ring is just a localization of $\Z$ at $\prod(p_i) = (\prod p_i)$. This
corresponds to affine subscheme of $\Spec \Z$ consisting of points
${(p_i)}$.

\paragraph{Ex. 2.3}
Use the formula for $\phi(n)$ to give a proof that there are infinitely many primes.

Let $p_1, \ldots, p_t$ be all primes. Consider $n = \prod p_i$. Then
$\phi(n) = n\prod(1-1/p_i) = n \prod(p_i - 1)/p_i = \prod(p_i -
1)$. Since 3 is prime, $\phi(n) > 1$, but this means that there exists
$1 \leq k < n$ that is relatively prime to $n$, but this is
impossible, as any of its prime factors must also be a factor of $n$.

\paragraph{Ex. 2.4}
If $a$ is a nonzero integer, then for $n > m$ show that $(a^{2^n} + 1, a^{2^m} + 1) = 1$ or $2$
depending on whether $a$ is odd or even.

First we'll prove that if a prime $p$ divides $a^{2^m} + 1$, it must
also divide $a^{2^n} -1$ for all $n > m$. This is simple induction:
for $n = m+1$, we have $p|a^{2^m} + 1|(a^{2^m}+1)(a^{2^m}-1) =
(a^{2^m})^2 - 1 = a^{2^{m+1}} - 1$. The induction step is similar.

Thus, if $p$ divides both $a^{2^n} + 1, a^{2^m} + 1$ for $n > m$, it
also divides $a^{2^n} - 1$ by reasoning above, so it must divide the
difference $a^{2^n} + 1 - (a^{2^n} - 1) = 2$. This means that $p$ must
be $2$. It is easy to see that this will be the case whenever $a$ is
odd -- in that case, both $a^{2^n} + 1, a^{2^m} + 1$ will be even.

\paragraph{Ex. 2.5}
Use the result of Ex. 2.4 to show that there are infinitely many primes.

Clearly, since $(2^{2^n} + 1, 2^{2^m} + 1) = 1$, all of the prime
factors of $2^{2^n} + 1$ are different from all of the prime factors
of $2^{2^m} + 1$ for $n \ne m$ Since there are infinitely many numbers
of the form $2^{2^n} + 1$, there must be infinitely many primes.

\paragraph{Ex. 2.6}
For a rational number $r$ let $[r]$ be the largest integer less than
or equal to $r$, e.g., $[1/2] = 0$, $[2] = 2$, and $[3+1/3] =
3$. Prove $ord_p n! = [n/p] + [n/p^2] + [n/p^3] + \ldots$.

Since $n!$ is a product of $1, 2, 3, \ldots, n$, every $p$-th integer
contribute a factor of $p$ to the whole product, which correspond to
the $[n/p]$ summand in the formula, every $p^2$-th factor contributes
another $p$ factor to the product, which corresponds to $[n/p^2]$
summand, etc.

\paragraph{Ex. 2.7}
Deduce from Ex. 2.6 that $ord_p n! \leq n/(p - 1)$ and that
$\sqrt[n]{n!} \leq \prod_{p|n!} p^{1/(p-1)}$.

We have:
\begin{equation}
ord_p n! = [n/p] + [n/p^2] + [n/p^3] + \ldots \leq n/p + n/p^2 +
\ldots = \frac{n}{p}\sum_{i = 0} 1/p^i = \frac{n}{p}\frac{1}{1-1/p} = \frac{n}{p-1}
\end{equation}

The inequality $\sqrt[n]{n!} \leq \prod_{p|n!} p^{1/(p-1)}$ is
equivalent to $n! \leq \prod_{p|n!} p^{n/(p-1)}$, which is easily
derived by using the previous inequality to an equality:

\begin{equation}
  m = \prod_{p|m} p^{ord_p m}
\end{equation}

\paragraph{Ex. 2.7}
Use Exercise 7 to show that there are infinitely many primes.

Let $p_1, \ldots, p_n$ be all primes. Then $\sqrt[n]{n!} \leq
\prod_{p|n!} p^{1/(p-1)} \leq \prod_i p_i^{1/(p_i - 1)}$, which is
independent of $n$. This implies that $\sqrt{n}{n!}$ is a bounded
sequence. However, $(n!)^2 \geq n^n$ -- this is seen by noting that
$(n!)^2 = \prod_{1 \leq i \leq n} i(n-i+1)$, and for $1 \leq i \leq
n$, $i(n-i+1) \geq n$ - indeed, a quadratic function $-i^2 + i(n+1)$
attains maximum for $i = (n+1)/2$, and is monotonically decreasing in
both directions, while still being no smaller than $n$ for both $i =
1$ and $i = n$.

Therefore, $\sqrt[n](n!) \geq \sqrt{n}$, but $\sqrt{n}$ is unbounded,
which is a contradiction with boundedness of $\sqrt[n](n!)$.

\paragraph{Ex. 2.8}
A function on the integers is said to be multiplicative if $f(ab) =
f(a)f(b)$. whenever $(a, b) = 1$. Show that a multiplicative function
is completely determined by its value on prime powers.

Trivial.

\paragraph{Ex. 2.9}
If $f(n)$ is a multiplicative function, show that the function $g(n) =
\sum_{d|n} f(d)$ is also multiplicative.

We'll prove a stronger theorem, that is, if $f$ and $g$ are
multiplicative functions, then their Dirichlet product is also a
multiplicative function. 

Indeed, for $(a, b) = 1$:
\begin{equation}
  (f \circ g(a)) \cdot (f \circ g(b)) = \left(\sum_{d_1 d_2 = a}
  f(d_1)g(d_2)\right)\left(\sum_{d_3 d_4 = b} f(d_3)g(d_4)\right) =
  \sum_{d_1 d_2 = a, d_3 d_4 = b} f(d_1 d_3) g(d_2 d_4)
\end{equation}

Now we just need to convince ourselves that this is equivalent to
$\sum_{u_1 u_2 = ab} f(u_1)g(u_2)$, but this is clear: since $(a, b) =
1$, if $u_1 u_2 = ab$, $u_1$ can be uniquely factored into $d_1 d_3$
such that $d_1 | a, d_3 | b$, and same for $u_2$.

\paragraph{Ex. 2.10}
Show that $\phi(n) = n \sum_{d|n}\mu(d)/d$ by first proving that
$\mu(d)/d$ is multiplicative and then using Ex. 2.9 and 2.10.

The function $\mu{d}/d$ is multiplicative, as it's a pointwise product
of two multiplicative functions, $\mu(d)$ and $1/d$. Therefore, by
Ex. 2.10, $\sum_{d|n}\mu{d}/d$ is also multiplicative, and so is
$n\sum_{d|n}\mu{d}/d$, as a pointwise product of multiplicative
functions (obviously $n$ is multiplicative).  Let $f(n) = n
\sum_{d|n}\mu{d}/d$. As $f$ is multiplicative, it is fully determined
by its values on prime powers. If we show that $f(p^n) = \phi(p^n)$,
it implies that $f = \phi$.

The only $1 \leq i \leq p^n$ that aren't relatively prime with $p$ are
multiples of $p$. These are $p, 2p, 3p, \ldots, p^{n-1} p^n$. There
are exactly $p^{n-1}$ elements on this list, so $\phi(p^n) = p^n -
p^{n-1}$.

On the other hand, $f(p^n) = p^n \sum_{d|p^n}\mu(d)/d$. The only
$d|p^n$ such that $\mu(d) \ne 0$ is $d = 1$ and $d = p$. Thus, $f(p^n)
= p^n \sum_{d|p^n}\mu{d}/d = p^n(1 - 1/p)$, and so $f(p^n) =
\phi(p^n)$.

\paragraph{Ex. 2.12}
Find formulas for $f(n) = \sum_{d|n} \mu(d)\phi(d)$, $g(n) = \sum_{d|n}
\mu(d)^2\phi(d)^2$, and $h(n) = \sum_{d|n} \mu(d)/\phi(d)$.

All of the functions are multiplicative, by Ex. 2.9. Let's determine their values on prime powers.

We have:
\begin{equation}
  f(p^n) = \sum_{p^k|p^n} \mu(p^k) \phi(p^k) = \phi(1) - \phi(p) = -1 - p + 1 = -p
\end{equation}
Thus $f(n) = (-1)^k \prod_{i=1}^k p_i$, where $p_i$ are distinct prime factors of $n$.

\begin{equation}
  g(p^n) = \sum_{p^k|p^n} \mu(p^k)^2 \phi(p^k)^2 = \phi(1)^2 + \phi(p)^2 = 1 + p^2 - 2p + 1 = p^2 - 2p + 2
\end{equation}
Formula for $g(n)$ easily follows, but doesn't seem to be interesting.

\begin{equation}
  f(p^n) = \sum_{p^k|p^n} \mu(p^k)/ \phi(p^k) = 1/\phi(1) - 1/\phi(p)
  = -1 - \frac{1}{p - 1} = \frac{-p}{p-1}
\end{equation}

\paragraph{Ex. 2.13}
Let $\sigma_k(n) = \sum_{d|n} d^k$ . Show that $\sigma_k(n)$ is
multiplicative and find a formula for it.

It is clearly multiplicative, since $f(n) = n^k$ is multiplicative. We have:

\begin{equation}
  \sigma_k(p^n) = \sum_{d|p^n} d^k = 1 + p^k + (p^2)^k + \ldots + (p^n)^k = 1 + p^k + (p^k)^2 + \ldots + (p^k)^n = \frac{1-(p^k)^{n+1}}{1-p^k}
\end{equation}

\paragraph{Ex. 2.14}
If $f(n)$ is multiplicative, show that $h(n) = \sum_{d|n} \mu(n/d)f(d)$ is also multiplicative.

It follows from our solution to Ex. 2.9, as $\mu$ is multiplicative.

\paragraph{Ex. 2.15}
Show that
\begin{enumerate}[a]
  \item $\sum_{d|n} \mu(n/d)\nu(d) = 1$ for all n.
  \item $\sum_{d|n} \mu(n/d) \sigma(n) = n$ for all n.
\end{enumerate}

Both of the left hand sides are multiplicative functions of $n$, so it's enough to determine their value on prime powers. We have:

\begin{equation}
  \sum_{d|p^n} \mu(p^n/d)\nu(d) = \mu(p^n/p^{n-1}) \nu(p^{n-1}) +
  \mu(p^n/p^n) \nu(p^n) = \mu(p)\nu(p^{n-1}) + \mu(1) \nu(p^n) = -n +
  n+1 = 1
\end{equation}
since for $d$ other than $p^{n-1}$ and $p^n$, $\mu(p^n/d)$ is 0.

For (b),

\begin{eqnarray}
  \sum_{d|p^n} \mu(p^n/d)\sigma(d) &=& \mu(p^n/p^{n-1}) \sigma(p^{n-1})
  + \mu(p^n/p^n) \sigma(p^n) \\
  &=& \mu(p)\sigma(p^{n-1}) + \mu(1) \sigma(p^n) \\
  &=& -\frac{p^n -1}{p-1} + \frac{p^{n+1} -1}{p-1} \\
  &=& \frac{1 - p^n -1 + p^{n+1}}{p-1} = \frac{p^n(p-1)}{p-1} \\
  &=& p^n
\end{eqnarray}

\paragraph{Ex. 2.16}
Show that $\nu(n)$ is odd iff $n$ is a square.

This follows immediately from the formula:

\begin{equation}
  \nu(\prod p_i^{a_i}) = \prod(a_i+1)
\end{equation}

\paragraph{Ex. 2.17}
Show that $\sigma(n)$ is odd iff $n$ is a square or twice a square.

We have:
\begin{equation}
  \sigma(n) = \sigma(\prod p_i^{a_i}) = \prod_i\sum_{j=0}^{a_i}p^j
\end{equation}

For $\sigma(n)$ to be odd, it is necessary and sufficient that each of
the factors $\sum_{j=0}^{a_i}p^j$ is odd. For odd $p$, if $a_i$ even,
then $\sum_{j=0}^{a_i}p^j$ has odd number of odd summands, and
therefore is odd. For $p = 2$, the sum is always odd. The thesis now
follows.

\paragraph{Ex. 2.18}
Prove that $\phi(n)\phi(m) = \phi((n, m))\phi([n, m])$.

We have:
\begin{equation}
  \label{eq:phi-gcd-lcm}
  \phi([n, m]) = \phi(nm/(n,m)) = \phi(n/(n, m))\phi(m) = \phi(n)\phi(m/(n,m))
\end{equation}

Now $(n, m)$ must be relatively prime with one of $n/(n, m)$,
$m/(n,m)$ -- otherwise, if there was a prime divisor $p$ of $(n, m)$
that was also a prime divisor of both $n/(n, m), m/(n,m)$, it would
mean that $p(n, m)$ divides both $n$ and $m$, and so $(n, m)$ would
not be a greatest common divisor of $n$ and $m$. Without loss of
generality we can assume that $(n, m)$ is relatively prime with
$n/(n,m)$. Multiplying (\ref{eq:phi-gcd-lcm}) by $\phi((n, m))$ we
get:

\begin{equation}
  \phi((n, m))\phi([n, m]) = \phi((n, m))\phi(n/(n, m))\phi(m) =
  \phi((n,m) \cdot n/(n, m))\phi(m) = \phi(n)\phi(m)
\end{equation}
which is what we wanted to show.

\paragraph{Ex. 2.19}
Prove that $\phi(nm)\phi((n, m)) = (n, m)\phi(n)\phi(m)$.

\paragraph{Ex. 2.20}
Prove that $\prod_{d|n} d = n^{\nu(n)/2}$.

%% If $\nu(n)$ is odd, then $n = m^2$ by Ex. 2.16. The thesis is then
%% equivalent to $\prod_{d|m^2} d = (m^2)^{\nu(n)/2} = m^{\nu(n)}$. We use
%%   the Moebius inversion formula:
%% \begin{equation}
%%   n = \prod_{d|n} \prod_{e|d} e^{\mu(e)}
%% \end{equation}

\paragraph{Ex. 2.21}

Define $\land(n) = \log p$ if $n$ is a power of $p$ and zero
otherwise. Prove that $\sum_{d|n} \mu(n/d)\log d = \land (n)$.

Let $f(n) = \sum_{d|n} \land(d)$. Clearly, this is the same as $f(n) =
\sum_{p^k|n} \log p$, and there are exactly $ord_p n$ summands equal
to $\log p$. Thus, $f(n) = \log \prod_{p^k|n} p = \log \prod_{p|n}
p^{ord_p n} = \log n$. The equality now follows from Moebius inversion
formula.

\paragraph{Ex. 2.22}
Show that the sum of all the integers $t$ such that $1 \leq t \leq n$ and $(t,
n) = 1$ is $\frac{1}{2} n \phi(n)$.

Let $f(n)$ be the sum in question. Consider fractions $1/n, 2/n,
\ldots, n/n$, and reduce them to lower terms. The possible
denominators are all $d|n$. Consider the sum of numerators of
fractions with denominators equal to $d$. These are all $1 \leq t \leq
d$ with $(t, d) = 1$, and so the sum will be equal to $f(d)$. They all
come from some fractions $k/n$ by dividing the numerator and
denominator by $n/d$, so the sum of all $k$-s in the fractions $k/n$
that reduce to a fraction with denominator $d$ will be exactly $n/d
f(d)$. Therefore, the following equality holds:

\begin{equation}
  \sum_{d|n} \frac{n}{d} f(d) = \frac{n(n+1)}{2}
\end{equation}
which is equivalent to:
\begin{equation}
  \sum_{d|n} \frac{f(d)}{d} = \frac{n+1}{2}
\end{equation}

We apply Moebius inversion formula:
\begin{equation}
  \frac{f(n)}{n} = \sum_{d|n}\mu\left(\frac{n}{d}\right) \frac{d+1}{2} =
  \frac{1}{2}\left(\sum_{d|n}\mu\left(\frac{n}{d}\right)d +
  \sum_{d|n}\mu\left(\frac{n}{d}\right)\right)
\end{equation}

The second sum is 0, so we are left with proving that
$\sum_{d|n}\mu\left(\frac{n}{d}\right)d = \phi(n)$, but this follows
instantly by applying Moebius inversion formula to the equality $n =
\sum_{d|n} \phi(d)$.

\paragraph{Ex. 2.23}
Let $f(x) \in \Z[x]$ and let $\psi(n)$ be the number of $f(j), j = 1,2, \ldots, n$,
such that $(f(j), n) = 1$. Show that $\psi(n)$ is multiplicative and that
$\psi(p^t) = p^{t-1} \psi(p)$. Conclude that $\psi(n) = n \prod_{p|n} \psi(p)/p$.

Suppose $a, b$ are relatively prime, and $1 \leq j \leq ab$. Then
$f(j)$ is relatively prime to $ab$ if and only if it is relatively
prime to both $a$ and $b$. Clearly, $f(j)$ is relatively prime to $a$
iff $f(j) \mod a$ is relatively prime to $a$, but since $f$ is
polynomial, $f(j) \mod a = f(j \mod a)$. Since the same holds for $b$,
$f(j)$ is relatively prime to $ab$ if and only if $f(j \mod a)$ is
relatively prime to $a$, and $f(j \mod b)$ is relatively prime to
$b$. Since $f(0) = f(a) \mod a$, there are exactly $\psi(a)\psi(b)$
numbers $1 \leq j \leq ab$ such that $f(j)$ is relatively prime to
$ab$ -- by the reasoning above, each such $j$ gives us a pair $(j \mod
a, j \mod b)$ (where $\mod$ are taken to be in $1, \ldots a$ and $1,
\ldots, b$ instead of $0, \ldots, a-1$ and $0, \ldots, b-1$) such that
$(f(j \mod a), a) = 1, (f(j \mod b), b) = 1$, and each pair $(r, s), 1
\leq r \leq a, 1 \leq s \leq b, (f(r), a) = 1, (f(s), b) = 1$ gives us
by Chinese remainder theorem a number $j$ in $1, \ldots, ab$ such that
$(f(j), ab) = 1$.

Now let $n = p^t$. For $1 \leq j \leq p^t$ we have $(f(j), p^t) =
(f(j), p) = (f(j) \mod p, p) = (f(j \mod p), p)$. Thus $(f(j), p^t) =
1$ iff $(f(j \mod p), p) = 1$. For $1 \leq j' \leq p$ there are
exactly $p^{t-1}$ different $j$, $1 \leq j \leq p^t$ such that $j \mod
p = j'$. Thus, $\psi(p^t) = p^{t-1} \psi(p)$.

Now, since $\psi$ is multiplicative, for $n = \prod_i p_i^{a_i}$ we
have $\psi(n) = \psi(\prod_i p_i^{a_i})= \prod_i \psi(p_i^{a_i}) =
\prod_i p^{a_i - 1} \psi(p) = \prod_i p^{a_i} \psi(p_i)/p_i = n
\prod_i \psi(p_i)/p_i$.

\paragraph{Ex. 2.24}
Supply the details to the proof of Theorem 3.

Since the Theorem 3 doesn't have any details left to supply, I assume
that the author means Theorem 4 here:

Theorem 4. $\sum q^{-\deg p(x)}$ diverges, where the sum is over all monic
irreducibles $p(x)\in k[x]$.

In the text authors show that $\sum q^{-\deg f(x)}$ diverges, and
$\sum q^{-2\deg f(x)}$ converges, where the sum is taken over all
monic polynomials. We now need to show that the same holds when one
takes the sum over only irreducibles.

Let $p_1, p_2, \ldots, p_{l(n})$ be all monic irreducible with degree
no larger than $n$, and define:

\begin{equation}
  \lambda(n) = \prod_{i=1}^{l(n)} \left(1-q^{-\deg(p_i)}\right)^{-1}
\end{equation}

Letting $l(n) = l$, we have:
\begin{eqnarray}
  \lambda(n) &=& \prod_{i=1}^{l} \sum_{j = 0}^\infty q^{-a_j
    \deg(p_i)} = \prod_{i=1}^{l} \sum_{j = 0}^\infty
  q^{-\deg(p_i^{a_j})} \\ &=& \sum \left(q^{\deg(p_1^{a_1})}
  q^{\deg(p_2^{a_2})} \ldots
  q^{\deg\left(p_{l}^{a_{l}}\right)}\right)^{-1} \\ &=& \sum
  \left(q^{\deg(p_1^{a_1})+\deg(p_2^{a_2})+ \ldots
    +\deg\left(p_{l}^{a_{l}}\right)}\right)^{-1} \\ &=& \sum
  \left(q^{\deg\left(p_1^{a_1}p_2^{a_2}\cdots p_{l}^{a_{l}}\right)}\right)^{-1}
\end{eqnarray}
where the sum is taken over all $l$-tuples of nonnegative integers
$(a_1, \ldots, a_l)$. Clearly, every polynomial $f \in k[x]$ with
$\deg(f) \leq n$ can be written as $f = p_1^{a_1}p_2^{a_2}\cdots
p_{l}^{a_{l}}$ for some $a_1, \ldots, a_l$. Therefore $\lambda(n) >
\sum_{\deg(f) \leq n} q^{-\deg(f)}$, and the latter sum diverges as $n
\to \infty$, and so $\lambda(n) \to \infty$. Therefore there are
infinitely many irreducible polynomials in $k[x]$.

Consider $\log \lambda(n)$. We have:

\begin{eqnarray}
  \log\lambda(n) &=&= -\sum_{i=1}^{l}
  \log\left(1-q^{-\deg(p_i)}\right) = \sum_{i=1}^{l}
  \sum_{m=1}^\infty (m q^{m\deg(p_i)})^{-1} \\ &=& q^{-\deg(p_1)} +
  q^{-\deg(p_2)} + \ldots + q^{-\deg(p_l)} + \sum_{i=1}^{l}
  \sum_{m=2}^\infty (m q^{m\deg(p_i)})^{-1}
\end{eqnarray}

Now, $\sum_{m=2}^\infty (m q^{m\deg(p_i)})^{-1} < \sum_{m=2}^\infty
q^{-m\deg(p_i)} = q^{-2\deg(p_i)}(1-q^{-\deg(p_i)})^{-1} \leq
2q^{-2\deg(p_i)}$. Therefore:

\begin{eqnarray}
  \log\lambda(n) &=& q^{-\deg(p_1)} + q^{-\deg(p_2)} + \ldots +
  q^{-\deg(p_l)} + \sum_{i=1}^{l} \sum_{m=2}^\infty (m
  q^{m\deg(p_i)})^{-1} \\ &\leq&\sum_{i=0}^\infty q^{-\deg(p_i)} + \sum_{i=1}^\infty 2 q^{-2\deg(p_i)}
\end{eqnarray}

We know that $\sum_{\deg(f) \leq n} q^{-2\deg(f)}$ converges, where
the sum is taken over all monic $f$, so $\sum_{i=1}^\infty 2
q^{-2\deg(p_i)}$ also converges. Therefore, if $\sum_{i=0}^\infty
q^{-\deg(p_i)}$, $\log \lambda(n)$ would have been bounded, which is
not true, since $\lambda(n)$ diverges.

\paragraph{Ex. 2.25}
Consider the function $\zeta(s) = \sum_{n=1}^\infty 1/n^s$. $\zeta$
is called the Riemann zeta function. It converges for s > 1. Prove the
formal identity (Euler's identity) $\zeta(s) = \prod_p (1-1/p^s)^{-1}$.

We have:

\begin{equation}
  \prod_p (1-1/p^s)^{-1} = \prod_p \sum_{i=0}^\infty \frac{1}{(p^i)^s}
\end{equation}

For $n = \prod p_i^{a_i}$, $n^s = \prod (p_i^{a_i})^s$, so $1/n^s =
1/\prod (p_i^{a_i})^s$ is a summand of the product above.

\paragraph{Ex. 2.26}
Verify the formal identities:
\begin{enumerate}[a)]
\item $\zeta(s)^{-1} = \sum \mu(n)/n^s$
  
We have:
\begin{equation}
  \left(\sum \frac{\mu(n)}{n^s}\right) \cdot \zeta(s)  = \sum_n \sum_{d|n} \frac{\mu(d)}{d^s(n/d)^s} = \sum_n \frac{1}{n^s}\sum_{d|n} \mu(d) = 1
\end{equation}
since $\sum_{d|n} \mu(d) = 0$ for $n \ne 1$.

\item $\zeta(s)^{2} = \sum \nu(n)/n^s$

We have:
\begin{equation}
  \zeta(s)^2 = \sum_n \sum_{d|n} \frac{1}{d^s(n/d)^s} = \sum_n
  \frac{1}{n^s}\sum_{d|n} 1 =\sum_n \frac{\nu(n)}{n^s}
\end{equation}

\item $\zeta(s)\zeta(s-1) = \sum \sigma(n)/n^s$
  
  We have:
\begin{eqnarray}
  \zeta(s)\zeta(s-1) &=& \left(\sum_{n=1}^\infty 1/n^s\right)
  \left(\sum_{n=1}^\infty 1/n^{s-1}\right) \\ &=&
  \left(\sum_{n=1}^\infty 1/n^s\right)\left(\sum_{n=1}^\infty
  n/n^s\right) \\
  &=& \sum_n \sum_{d|n} \frac{d}{(n/d)^sd^s} = \sum_n
  \frac{1}{n^s} \sum_{d|n} d = \sum_n \frac{\sigma(n)}{n^s}
\end{eqnarray}
\end{enumerate}

\paragraph{Ex. 2.27}
Show that $\sum 1/n$, the sum being over square free integers,
diverges. Conclude that $\prod_{p < N} (1+1/p) \to \infty$ as $N \to
\infty$. Since $e^x > 1 + x$, conclude that $\sum_{p < N} 1/p \to
\infty$.

Any nonnegative $n$ can be written uniquely as $n = a b^2$, $a$ square free. Thus, we have:
\begin{equation}
  \sum_{n = 1}^\infty \frac{1}{n} = \sum_{a \, \textrm{square free}}
  \sum_{b = 1}^\infty \frac{1}{ab^2} = \sum_{a \, \textrm{square
      free}} \frac{1}{a} \sum_{b = 1}^\infty \frac{1}{b^2} = \frac{\pi^2}{6}\sum_{a \, \textrm{square free}} \frac{1}{a}
\end{equation}

Thus, if $\sum_{a \, \textrm{square free}} 1/a$ converged, so would
$\sum_n 1/n$, which is known to diverge.

Therefore, we have:
\begin{equation}
  \prod_{p < N}(1+\frac{1}{p}) = \sum_a \frac{1}{a}
\end{equation}
where sum is taken over all square free $a$ such that their prime
factos are smaller than $N$. Thus $\prod_{p < N}(1+\frac{1}{p}) \to
\sum_{a \, \textrm{square free}} 1/a = \infty$. But since $e^x > 1 + x$,

\begin{equation}
  \prod_{p < N}(1+\frac{1}{p}) \leq \prod_{p<N} e^{1/p} = e^{\sum_{p<N} 1/p}
\end{equation}

If $\sum_p 1/p$ converged, $\prod_p (1+\frac{1}{p})$ would
have been bounded by $e^{\sum_{p} 1/p}$, but we have just shown it
diverges.

\section{Chapter 3}

\paragraph{Ex. 3.1}
Show that there are infinitely many primes congruent to $-1$ modulo $6$.

All primes are congruent either to $1$ or $-1$ modulo $6$. Suppose
$p_1, \ldots, p_n$ is a list of all primes congruent to $-1$ modulo
$6$. Consider a number $6p_1 p_2 \cdots p_n - 1$. It is relatively
prime to all of $p_i$, so all of its prime factors must be congruent
to $1$ modulo $6$. But if it was the case, it would also have been
congruent to $1$ modulo $6$, but we see that it is congruent to $-1$
modulo 6.

\paragraph{Ex. 3.2}
Construct addition and multiplication tables for $\Z/5\Z, \Z/8\Z, and
\Z/10\Z$.

Skipped.

\paragraph{Ex. 3.3}
Let $abc$ be the decimal representation for an integer between $1$ and
$1000$. Show that $abc$ is divisible by $3$ iff $a + b + c$ is
divisible by $3$. Show that the same result is true if we replace $3$
by $9$. Show that $abc$ is divisible by $11$ iff $a - b + c$ is
divisible by $11$.  Generalize to any number written in decimal
notation.

Let $n = \sum_{i=0}^k a_i 10^i$ for $0 \leq a_i \leq 9$. Then $n$ is
divisible by $3$ iff $n \equiv 0 \pmod 3$. But $10 \equiv 1 \mod 3$,
so $n \equiv \sum_{i=0}^k a_i \pmod 3$. For divisibility by $9$ and
$11$ note that $10 \equiv 1 \pmod 9$ and $10 \equiv -1 \pmod {11}$.

\paragraph{Ex. 3.4}
Show that the equation $3x^2 + 2 = y^2$ has no solution in integers.

We have either $y^2 \equiv 0 \pmod 3$ or $y^2 \equiv 1 \pmod 3$, while
the left hand side is congruent to $2$ modulo $3$.

\paragraph{Ex. 3.5}
Show that the equation $7x^2 + 2 = y^3$ has no solution in integers.

The left hand side is congruent to $2$ mod $7$, while possible values
of $y^3 \pmod 7$ are $0^3 = 0$, $1^3 = 1$, $2^3 = 8 \equiv 1 \pmod 7$,
$3^3 = 27 \equiv 6 \pmod 7$, $4^3 = 16 \cdot 4 \equiv 2 \cdot 4 \equiv
1 \pmod 7$, $5^3 = 25 \cdot 5 \equiv 4 \cdot 5 \equiv 20 \equiv 6
\pmod 7$, and $6^3 = 36 \cdot 6 \equiv 1 \cdot 6 \equiv 6 \pmod 7$.

\paragraph{Ex. 3.6}
Let an integer $n > 0$ be given. A set of integers $a_1, \ldots,
a_{\phi(n)}$ is called a reduced residue system modulo $n$ if they are
pairwise incongruent modulo $n$ and $(a_i, n) = 1$ for all $i$. If
$(a, n) = 1$, prove that $aa_1 , aa_2, \ldots, aa_{\phi(n)}$ is again
a reduced residue system modulo $n$.

Let $b$ be inverse to $a$ modulo $n$, that is, $ab \equiv 1 \pmod n$
(it exists, because $(a, n) = 1 \pmod n$. Suppose $aa_i \equiv aa_j
\pmod n$. Then $ba a_i \equiv baa_j \pmod n$, so $a_i \equiv a_j \pmod
n$, and so $i = j$, thus $aa_1, aa_2, \ldots, aa_{\phi(n)}$ are
pairwise incongruent. As $(a,n) = 1$ and $(a_i, n) = 1$, we also have
$(aa_i, n) = 1$ -- otherwise, if $aa_i$ and $n$ had common prime
factor, it would have to be one of the factors of $a$ or $a_i$, but
they have no common factors with $n$.

\paragraph{Ex. 3.7}
Use Ex. 2.6 to give another proof of Euler's theorem, $a^{\phi(n)}
\equiv 1 \pmod n$ for $(a, n) = 1$.

If $a_1, \ldots, a_{\phi(n)}$ and $b_1, \ldots, b_{\phi(n)}$ are two
reduced residue systems modulo $n$, there exists a unique bijection
$f$ such that $a_i \equiv b_{f(i)} \pmod n$ for all $i$ -- to find it,
note that $a_1 \mod n, \ldots, a_{\phi(n)} \mod n$ are all the
different integers between $1$ and $n$ relatively prime to $n$, and
same for $b_i$. Thus, we have:

\begin{equation}
  \prod_{i=1}^{\phi(n)} a_i \equiv \prod_{i=1}^{\phi(n)} b_{f(i)} = \prod_{i=1}^{\phi(n)} b_{i} \pmod n
\end{equation}

Now take $b_i = a a_i$. We have:
\begin{equation}
  \prod_{i=1}^{\phi(n)} a_i \equiv \prod_{i=1}^{\phi(n)} aa_i = a^{\phi(n)} \prod_{i=1}^{\phi(n)} a_i\pmod n
\end{equation}

Since all $a_i$ are invertible modulo $n$, their product also is, so
multiplying the equation above by the inverse of
$\prod_{i=1}^{\phi(n)} a_i$ we get the desired equality.

\paragraph{Ex. 3.8}
Let $p$ be an odd prime. If $k \in \{1, 2, \ldots, p - 1\}$, show that
there is a unique $b_k$ in this set such that $kb_k \equiv 1 \pmod
p$. Show that $k \ne b_k$ unless $k = 1$ or $k = p - 1$.

The existence and uniqueness of $b_k$ follow from the fact that
$\Z/p\Z$ is a field. Now assume that $k^2 \equiv 1 \pmod p$, or
equivalently $k^2 - 1 \equiv 0 \pmod p$. Then $(k-1)(k+1) \equiv 0
\pmod p$, but since $\Z/p\Z$ is an field, either $k-1 \equiv 0 \pmod
p$, or $k+1 \equiv 0 \pmod p$. First case corresponds to $k = 1$, and
the second to $k = p-1$.

\paragraph{Ex. 3.9}
Use Ex. 3.7 to prove that $(p - 1)! \equiv -1 \pmod p$.

Of course, authors here mean Ex. 3.8, and the desired inequality
follows immediately from it -- since every $k$ except of $k = p-1$ is
uniquely paired with corresponding $b_k$ such that $k b_k \equiv 1
\pmod p$, we can group factors of $(p-1)!$ other than p-1 into pairs
such that their product is congruent to $1$ modulo $p$. Thus $(p-1)!
\equiv p-1 \equiv -1 \pmod p$.

\paragraph{Ex. 3.10}
If $n$ is not a prime, show that $(n - 1)! \equiv 0 \pmod n$, except
when $n = 4$.

If $n$ is not a prime and not a square of a prime, it can be written
as $n = a b$ with $1 < a < b < n$. Both $a$ and $b$ are factors of
$(n-1)!$, but since $ab \equiv 0 \pmod n$, $(n-1)! \equiv 0 \pmod n$
also.

If $n$ is a square of a prime, $n = p^2$, and $n \ne 4$, then $1 < p <
2p < p^2$, so both $p$ and $2p$ are factors of $(p^2 - 1)!$, but $p
\cdot 2p = 2p^2 \equiv 0 \pmod {p^2}$.

If $n = 4$, $(n-1)! = 2 \cdot 3 = 6 \equiv 2 \pmod 4$.

\paragraph{Ex. 3.11}
Let $a_1, \ldots, a_{\phi(n)}$ be a reduced residue system modulo $n$
and let $N$ be the number of solutions to $x^2 \equiv 1 \pmod
n$. Prove that $a_1 \cdots a_{\phi(n)} \equiv (-1)^{N/2} \pmod n$.

All of $a_1, \ldots, a_{\phi(n)}$ are invertible modulo $n$, and their
inverse is also in this set. If $a_i^2 \equiv 1 \pmod n$, then also
$(-a_i)^2 \equiv 1 \pmod n$, and moreover, $a_i$ is not congruent to
$-a_i$ modulo $n$ -- otherwise, if $a_i \equiv -a_i \pmod n$, we'd
have $2 a_i \equiv 0 \pmod n$, which cannot happen, since $a_i$ is
invertible modulo $n$. Therefore, $a_i$ can be grouped into pairs --
if $a_i^2 \equiv 1$, we group $a_i$ with $-a_i$, and if $a_i^2 \not
\equiv 1$, we group it with its inverse, which is also in the reduced
residue system. Thus, we get $\phi(n)/2$ pairs. For pairs of the form
$\{a_i, a_i^{-1}\}$ the product $a_i a_i^{-1}$ is congruent to $1$
modulo $n$, and for pairs of the form $\{a_i, -a_i\}$ their product
is congruent to $-1$ modulo $n$. There are exactly $N/2$ pairs of the
second form, from which the desired congruence follows.

\paragraph{Ex. 3.12}
Let ${p \choose k} = \frac{p!}{k!(p-k)!}$ be a binomial coefficient,
and suppose $p$ is prime. If $1 \leq k \leq p-1$, show that $p$
divides ${p \choose k}$. Deduce $(a+b)^p \equiv a^p + b^p \pmod p$.

The divisibility is clear: since ${p \choose k} =
\frac{p!}{k!(p-k)!}$, $p$ divides the numerator, but none of the
factors in the denominator is divisible by $p$, as they all are
integers smaller than $p$.

The desired congruence now follows from the application of the
binomial theorem:

\begin{equation}
  (a+b)^p = \sum_{k=0}^p {p \choose k} a^k b^{p-k} \equiv a^p + b^p \pmod p
\end{equation}

The congruence holds, since ${p \choose k} a^k b^{p-k} \equiv 0 \pmod
p$, for $1 \leq k \leq p-1$, since ${p \choose k}$ is then divisible
by $p$.

\paragraph{Ex. 3.13}
Use Ex. 3.12 to give another proof of Fermat's theorem, $a^{p-1}
\equiv 1 \pmod p$ if $p$ does not divide $a$.

As $a$ is invertible modulo $p$, it is enough to prove that $a^p
\equiv a \pmod p$. We proceed by induction on $a$. The case $a = 1$ is
trivial. Assume that $(a-1)^p \equiv a-1 \pmod p$. By Ex. 3.12 we have:

\begin{equation}
  a^p = ((a-1)+1)^p \equiv (a-1)^p + 1 \equiv a-1 + 1 = a \pmod p
\end{equation}

\paragraph{Ex. 3.14}
Let $p$ and $q$ be distinct odd primes such that $p - 1$ divides $q -
1$. If $(n, pq) = 1$, show that $n^{q - 1} \equiv 1 \pmod {pq}$.

The congruence $n^{q - 1} \equiv 1 \pmod {pq}$ is equivalent to
$n^{q-1} - 1$ being divisible by $pq$, which in turns is equivalent to
being divisible by both $p$ and $q$, thus $n^{q - 1} \equiv 1 \pmod
{pq}$ iff both $n^{q - 1} \equiv 1 \pmod {p}$ and $n^{q - 1} \equiv 1
\pmod {q}$. The second congruence follows immediately from Euler's
theorem, while the first congruence follows from the fact that $q-1 =
a(p-1)$ for some $a$, so since by Euler theorem $n^{p-1} \equiv 1
\pmod p$, $n^{q-1} = n^{a(p-1)} = (n^{p-1})^a \equiv 1^a = 1 \pmod p$.

\paragraph{Ex. 3.15}
For any prime $p$ show that the numerator of $1+ \frac{1}{2} +
\frac{1}{3} + \ldots + \frac{1}{p}-1$ is divisible by $p$.

The theorem as stated is clearly wrong, e.g. for $p = 2$, $1 + 1/2 - 1
= 1/2$, for $p = 3$ we have $1 + 1/2 + 1/3 -1 = 5/6$. We'll show the
opposite theorem, namely that the numerator is \emph{not} divisible by
$p$.

We have:
\begin{equation}
  \frac{1}{2} + \frac{1}{3} + \ldots + \frac{1}{p} = \frac{p!/2 + p!/3
    + \ldots + p!/p}{p!}
\end{equation}

Note that $p!/k$ are integers, all of which are divisible by $p$,
except of the last one, which is not. Thus, their sum will not be
divisible by $p$.

\paragraph{Ex. 3.16}
Use the proof of the Chinese Remainder Theorem to solve the system $x \equiv 1 \pmod 7$,
$x \equiv 4 \pmod 9$, $x \equiv 3 \pmod 5$.

Skipped.

\paragraph{Ex. 3.17}
Let $f(x) \in \Z[x]$ and $n = p_1^{a_1} \cdots p_t^{a_t}$. Show that
$f(x) \equiv 0 \pmod n$ has a solution iff $f(x) \equiv 0 \pmod
{p_i^{a_i}}$ has a solution for $i = 1, \ldots, t$.

Follows from Ex. 3.13.

\paragraph{Ex. 3.18}
For $f \in \Z[x]$, let $N$ be the number of solutions to $f(x) \equiv
0 \pmod n$ and $N_i$ be the number of solutions to $f(x) \equiv 0
\pmod {p_i^{a_i}}$. Prove that $N = \prod N_i$.

Let $(b_1, \ldots, b_t)$ be a solution to a system $f(x) \equiv 0
\pmod {p_i^{a_i}}$. By Chinese remainder theorem there exists $x$ such
that $x \equiv b_i \pmod {p_i^{a_i}}$. We claim that $x$ is a solution
to $f(x) \equiv 0 \pmod n$. Indeed, this is the same as saying that
$f(x)$ is divisible by $p_i^{a_i}$ for $i = 1, \ldots, t$. But since
$x \equiv b_i \pmod {p_i^{a_i}}$, and $f$ is a polynomial, $f(x)
\equiv f(b_i) \equiv 0 \pmod {p_i^{a_i}}$, so $f(x)$ is divisible by
${p_i^{a_i}}$ for all $i$.

Thus, there's a one-to-one correspondence between tuples $(b_1,
\ldots, b_t)$ forming solutions to $f(x) \equiv 0 \pmod {p_i^{a_i}}$,
and $x$-s being the solutions to $f(x) \equiv 0 \pmod {n}$.

\paragraph{Ex. 3.19}
If $p$ is an odd prime, show that $1$ and $-1$ are the only solutions
to $x^2 \equiv 1 \pmod {p^a}$.

The given equation is equivalent to asking for $1 < x < p^a-1$ such
that $x^2 - 1$ is divisible by ${p^a}$. Now $x^2 - 1 = (x-1)(x+1)$,
and since $p > 2$, if one factor is divisible by $p$, the other factor
isn't, so the only way $(x-1)(x+1)$ is divisible by $p^a$ is that
either $p^a$ divides $x-1$, or it divides $x+1$. As $1 < x < p^a-1$,
this is impossible.

\paragraph{Ex. 3.20}
Show that $x^2 \equiv 1 \pmod {2^b}$ has one solution if $b = 1$, two
solutions if $b = 2$, and four solutions if $b \geq 3$.

The cases $b=1$ and $b=2$ are trivial, so assume that $b \geq 3$. We
ask for $1 \leq x \leq 2^b - 1$ such that $(x-1)(x+1)$ is divisible by
$2^b$, which is the same as asking that $ord_2(x-1)(x+1) = b$. We have
$(x-1, x+1) = 2$, so necessarily $ord_2(x-1)(x+1) \leq
\max(ord_2(x-1), ord_2(x+1)) + 1$.  For $1 < x < 2^b - 1$,
$\max(ord_2(x-1), ord_2(x+1)) + 1 < b$ except for $x$ such that $x-1 =
2^{b-1}$ or $x+1 = 2^{b-1}$. These are 2 solutions, and since $b > 3$,
they aren't equal to $x = 1$ or $x = -1$.

\paragraph{Ex. 3.21}
Use Ex. 3.18-3.20 to find the number of solutions to $x^2 \equiv 1 \pmod n$.

Write $n = 2^a p_1^{a_1} \cdots p_t^{a_t}$ for $p_1 \ne 2$. By
Ex. 3.18, the number of solutions is equal the number of solutions
modulo $2^a$ times the numbers of solutions modulo $p_i^{a_i}$. By
Ex. 3.19 and 3.20, this is equal to $2^t$ for $a = 0, 1$, $2^{t+1}$
for $a = 2$ and $2^{t+2}$ for $a \geq 3$.

\paragraph{Ex. 3.22}
Formulate and prove the Chinese Remainder Theorem in a principal ideal
domain.

Let $A$ be a PID. Let $m_1, \ldots, m_n$ be such that $(m_i, m_j) =
A$. Then $A/(m_1 \cdots m_n) \simeq A/(m_1) \times \cdots \times
A/(m_n)$.

The proof is omitted, as it is exactly the same as proof for $\Z$.

\paragraph{Ex. 3.23}
Extend the notion of congruence to the ring $\Z[i]$ and prove that $a
+ bi$ is always congruent to $0$ or $1$ modulo $1 + i$.

Let $\pi$ be a prime element of $\Z[i]$. Then for $x, y \in \Z[i]$,
$x$ is said to be congruent modulo $\pi$ iff $x-y$ is divisible by
$\pi$.

Note that $(1+i)^2 = 1 + 2i - i^2 = 2i$, so $2i \equiv 0 \pmod
{1+i}$. Now take any $a + bi \in \Z[i]$. If $a-b$ is even, then:

\begin{equation}
  a+bi \equiv a+bi + \frac{a-b}{2} \cdot 2i \equiv a+ai \equiv a(1+i)
  \equiv 0 \pmod {1+i}
\end{equation}

Alternatively, if $a-b$ is odd, then $a-1+bi \equiv 0 \pmod {1+i}$ by
the reasoning above, so $a-1+bi \equiv 0 \pmod {1+i}$, and so $a + bi
\equiv 1 \pmod {1+i}$.

\paragraph{Ex. 3.24}
Extend the notion of congruence to the ring $\Z[\omega]$ and prove
that $a + b\omega$ is always congruent to $-1$, $0$ or $1$ modulo $1 -
\omega$.

Note that $(1-\omega)(1+\omega)(1-\omega) = (1 - \omega^2)(1-\omega) =
(1 - \bar{\omega})(1-\omega) = 1 - \omega - \bar{\omega} +
\omega\bar{\omega} = 1 - (2 \Re \omega) + |\omega|^2 = 1 + 1 + 1
=3$. Reasoning now is similar to the reasoning in Ex. 3.23.

\paragraph{Ex. 3.25}
Let $\lambda = 1 -\omega \in \Z[\omega]$. If $\alpha \in \Z[\omega]$
and $\alpha \equiv 1 \pmod \lambda$, prove that $\alpha^3 \equiv 1
\pmod 9$.

If $\alpha \equiv 1 \pmod \lambda$, then $\alpha = 1 + \beta \lambda$,
for some $\beta \in \Z[\omega]$. Then $\alpha^3 = 1 + 3 \beta \lambda
+ 3 \beta^2 \lambda^2 + \beta^3 \lambda^3$. Substituting $\lambda = 1
- \omega$, we get:

\begin{eqnarray}
  \alpha^3 - 1 &=& 3 \beta \lambda + 3 \beta^2 \lambda^2 + \beta^3
  \lambda^3 \\ &=& 3 \beta - 3 \beta \omega + 3 \beta^2 - 6 \beta^2
  \omega + 3 \beta^2 \omega^2 - 3 \beta^3 \omega + 3 \beta^3 \omega^2
  \\ &=& 3 \beta(1 - \omega + \beta - 2 \beta \omega + \beta \omega^2
  - \beta^2 \omega + \beta^2 \omega^2) \\ &=& 3\beta(1 - 2 \beta
  \omega + \beta^2 \omega^2 - \omega + \beta \omega^2 + \beta -
  \beta^2 \omega) \\ &=& 3 \beta((1 - \beta \omega)^2 - \omega(1 -
  \beta \omega) + \beta(1 - \beta \omega)) \\ &=& 3 \beta (1 - \beta
  \omega)(1 - \beta \omega - \omega + \beta) \\ &=& 3 \beta (1 - \beta
  \omega)(1 + \beta)(1- \omega)
\end{eqnarray}

It is now enough to show that $\beta (1 - \beta \omega)(1 + \beta)(1-
\omega)$ is divisible by $3$, so we can work modulo $3$. We let $\beta
= a + b \omega$, and after reducing modulo $3$ we can assume that $-1
\leq a, b \leq 1$. We have:

\begin{eqnarray}
  \label{eq:fukken-9}
   \beta (1 - \beta \omega)(1 + \beta)(1- \omega) &=& 3(a +
   b\omega)(1-(a + b \omega)\omega)(1 + a + b \omega)(1-\omega) \\ &=&
   (a + b\omega)(1-a \omega - b \omega^2)(1 + a + b \omega)(1-\omega)
   \\ &=& (a + b\omega)(1+b +(b-a) \omega)(1 + a + b \omega)(1-\omega)
\end{eqnarray}

We need to prove that the expression above is divisible by 3. If $a =
b = 0$, it is trivial so we can assume that at least one of $a, b$ is
nonzero.  Note that $-\omega^2 (1-\omega)^2 = -\omega^2 (1 - 2 \omega
+ \omega^2) = - \omega^2 + 2 - \omega = 3 - 1 - \omega - \omega^2 =
3$, so $(1-\omega)^2$ is divisible by $3$.

If $a = 1, b = -1$, $a + b \omega = 1 - \omega$, so we get two factors
of $1-\omega$ in (\ref{eq:fukken-9}). If $a = 1, b = 0$, then $(1+b
+(b-a) \omega) = 1 - \omega$. If $a = 1, b = 1$, then $(1+a+b\omega) =
2 + \omega$ (note that this is the same case as $-1 + \omega$). We can
multiply it by $\omega$ (which is invertible) to get $\omega(2 +
\omega) = 2 \omega + \omega^2 = 2 - 1 - \omega = 1 - \omega$.

If $a = -1$, $b = -1$, then $1 + b + (b - a)\omega = 0$, and so the
whole expression is 0, which is divisible by 3. If $a = -1, b = 0$,
then $1 +b + (b-a)\omega = 1 - \omega$. The case $a = -1, b = 1$ has
been dealt with above.

In all cases, we get the expression to equal 0, to have factor of
$(1-\omega)^2$, or to have a factor associated to $(1-\omega)^2$
through multiplication by $\omega$, and so we are done.

This seems to be like a very brute force approach to the problem
above, and so I'm looking for a cleaner approach.

\paragraph{Ex. 3.26}
Use Ex. 3.25 to show that $\xi, \eta, \zeta$ are not zero and $\xi^3 +
\eta^3 + \zeta^3 = 0$, then $\lambda$ divides at least one of the
elements $\xi, \eta, \zeta$.

Assume that $\lambda$ divides none of $\xi, \eta, \zeta$. We aim to
show contradiction. By Ex. 3.24 we know that each of $\xi, \eta,
\zeta$ must be congruent to $-1$ or $1$ modulo $\lambda$. Clearly
$\alpha^3 \equiv \alpha \pmod \lambda$ for any $\alpha$, so since
$\xi^3 + \eta^3 + \zeta^3 = 0$, and since $-1, 0, 1$ are different
classes modulo $\lambda$, necessarily we have $\xi \equiv \eta \equiv
\zeta \equiv \pm 1 \pmod \lambda$. Multiplying by $-1$, which is
invertible, we can assume that they all are congruent to $1$ modulo
$\lambda$. Then, Ex. 3.25 tells us that $\xi^3 + \eta^3 + \zeta^3
\equiv 3 \pmod 9$, but $\xi^3 + \eta^3 + \zeta^3 = 0$ and $0 \not
\equiv 3 \pmod 9$, which is a contradiction.

\section{Chapter 4}

\paragraph{Ex. 4.1}
Show that $2$ is a primitive root modulo $29$.

We are asked to show that $2$ generates the multiplicative group of
the field $\Z/29\Z$, which has order $28$. Since $28 = 4 \cdot 7$,
it's enough that we show that its orderd is not either $2$, $4$, $7$
or $14$.

Note that $2^2 \equiv 4 \pmod {29}$, $2^4 \equiv 16 \pmod {29}$, $2^7
\equiv 4 \cdot 2^5 \equiv 4 \cdot 32 \equiv 4 \cdot 3 \equiv 12 \pmod
{29}$, and $2^{14} \equiv (2^7)^2 \equiv 12^2 \equiv 144 \equiv 28 \pmod
{29}$.

\paragraph{Ex. 4.2}
Compute all primitive roots for $p = 11, 13, 17$, and $19$.

Skipped.

\paragraph{Ex. 4.3}
Suppose that $a$ is a primitive root modulo $p^n$, $p$ an odd
prime. Show that $a$ is a primitive root modulo $p$.

Assume that $a$ is not a primitive root modulo $p$, that is, $a^k
\equiv 1 \pmod p$ for some $k | p-1$, $k < p-1$. By repeated
application of Lemma 3, from $a^k \equiv 1 \pmod p$ we get
$(a^k)^{p^{n-1}} \equiv 1^{p^{n-1}} \pmod {p^n}$, so order of $a$ is
no larger than $k p^{n-1}$, which is smaller than $\phi(p^n) =
p^{n-1}(p-1)$. Therefore, $a$ cannot be a primitive root modulo $p^n$.

\paragraph{Ex. 4.4}
Consider a prime $p$ of the form $4t + 1$. Show that $a$ is a
primitive root modulo $p$ iff $-a$ is a primitive root modulo $p$.

Assume that $a$ is a primitive root modulo $p$. Suppose $k$ is a
smallest integer such that $(-a)^k \equiv 1 \pmod p$. We need to show
that $k = p - 1$. Note that $(-a)^k \equiv (-1)^k a^k \equiv 1 \pmod
p$, and since $a$ is a primitive root modulo $p$, if $k < p-1$, then
also $a^k \not \equiv 1$, so we must have $(-1)^k \equiv -1 \pmod p$,
that is, $k$ is odd. Since $p = 4t + 1$, $p-1$ is divisible by $4$,
and so $2k < p-1$. But then $a^{2k} = (-1)^{2k}a^{2k} = (-a)^{2k} =
((-a)^k)^2 \equiv 1^2 \pmod p$, and so $a$ is not a primitive root
modulo $p$.

\paragraph{Ex. 4.5}
Consider a prime $p$ of the form $4t + 3$. Show that $a$ is a
primitive root modulo $p$ iff $-a$ has order $(p - 1)/2$.

Assume that $a$ is a primitive root modulo $p$. Let $k$ be the order
of $-a$ modulo $p$. Now if $k < (p-1)/2$, reasoning as above, we
conclude that $k$ must be odd. Also $2k < p-1$, and reasoning as
above, $a$ is not a primitive root modulo $p$. Therefore, we get $k
\geq (p-1)/2$.

On the other hand, if $-a$ had order strictly larger than $(p-1)/2$,
the only remaining option is $p-1$, that is, $-a$ is a primitive root
modulo $p$. It means that $(-a)^{(p-1)/2} \not \equiv 1 \pmod p$. But
$(p-1)/2 = 2t + 1$, so $(-1)^{2t+1} a^{2t+1} \not \equiv 1 \pmod p$,
that is, $-a^{2t+1} \not \equiv 1 \pmod p$. But $(a^{2t+1})^2 =
a^{p-1} \equiv 1 \pmod p$, and so since we are working in a field,
$a^{2t+1}$ is congruent to $1$ or $-1$ modulo $p$. Since $a$ is
primitive root modulo $p$, it must be congruent to $-1$ modulo $p$,
but then $-a^{2t+1} \equiv 1 \pmod p$, which is a contradiction.

In the other direction, assume tha $-a$ has order $(p - 1)/2$. It
means that $(-a)^{2t+1} \equiv 1 \pmod p$, that is, $a^{2t+1} \equiv
-1 \pmod p$. Order of $a$ modulo $p$ must divide $p-1$, and the
congruence above show that it does not divide $2t+1 = (p-1)/2$, so it
must be equal to $p-1$.

\paragraph{Ex. 4.6}
If $p = 2^{2^n} + 1$ is a Fermat prime, show that $3$ is a primitive root
modulo $p$.

If the order of $3$ is smaller than $p-1$, it must divide $(p-1)/2$,
so it is enough to show that $3^{(p-1)/2} \equiv -1 \pmod p$, which is
equivalent to saying that $3$ is not a square modulo $p$. Since $-1$
is a square modulo $p$ (as can be seen by Wilson theorem for any $p =
4t +1$), it follows that $-3$ is a non-square too, so it is enough to
show that $-3$ cannot be square modulo $p$. Suppose it is, that is,
$x^2 \equiv -3 \pmod p$ for some $x$. We can assume that $x$ is odd --
if $x = 2y$ and $(2y)^2 \equiv -3 \pmod p$, then we just replace $2y$
with $2y+p$, which is odd. Thus, $x = 2y+1$ for some $y$. Then if
$(2y+1)^2 \equiv -3 \pmod p$, then $4y^2 + 4y + 4 \equiv 0 \pmod p$,
but $4$ is invertible modulo $p$, so $y^2 + y + 1 \equiv 0 \pmod
p$. It follows that $y^3 \equiv 1 \pmod p$, but since order of $y$
must divide $p-1 = 2^n$, $y \equiv 1 \pmod p$. Thus $3^2 \equiv -3
\pmod p$, that is, $12 \equiv 0 \pmod p$, and since $12 =
2\cdot2\cdot3$, one of $2$ of $3$ is congruent to $0$ modulo $p$,
which is only true if $p = 3$ (since $p$ is a Fermat prime). But $p
\ne 3$ is a hidden assumption in th exercise, since in that case $3$
clearly cannot be a primitive root modulo $3$.

\paragraph{Ex. 4.7}
Suppose that $p$ is a prime of the form $8t + 3$ and that $q = (p -
1)/2$ is also a prime.  Show that $2$ is a primitive root modulo $p$.

By Ex. 4.5, it is enough to show that $-2$ has order $(p-1)/2 =
q$. Let $k$ be the order of $-2$. Since $k$ must divide $p-1$, and
$p-1 = 2q$, so $k$ must be $2$, $q$, or $2q$. 

If the order is $2$, then $(-2)^2 \equiv 1 \pmod p$, so $3 \equiv 0
\pmod p$, and so $p = 3$, but then $q = (p-1)/q$ is not prime.

If the order of $-2$ is $2q$, then $-2$ is a quadratic nonresidue. As
$p = 8t+3$, $-1$ is also a quadratic nonresidue. Therefore $2$ is a
quadratic residue modulo $p$, but this is only true for $p$ of the
form $8t+1$, $8t+7$ (proved in the next chapter), which is a
contradiction.

\paragraph{Ex. 4.8}
Let $p$ be an odd prime. Show that $a$ is a primitive root modulo $p$
iff $a^{(p-1)/q} \not \equiv 1 \pmod p$ for all prime divisors $q$ of $p - 1$.

One direction is clear: if $a^{(p-1)/q} \equiv 1 \pmod p$ for some
$q$, then the order of $a$ is smaller than $p-1$.

In the other direction, assume that the order of $a$ is smaller than
$p-1$, say $a^k \equiv 1 \pmod p$ for some $k < p-1$. The order $k$
must divide $p-1$, so write $p-1 = k t$ for some $t > 1$. Let $q$ be a
prime divisor of $t$, so that $p-1 = k t' q$ for some $t'$. Then:

\begin{equation}
  a^{(p-1)/q} \equiv a^{kt'} \equiv (a^k)^{t'} \equiv 1^{t'} \equiv 1 \pmod p
\end{equation}

\paragraph{Ex. 4.9}
Show that the product of all the primitive roots modulo $p$ is
congruent to $(-1)^{\phi(p-1)}$ modulo $p$.

Let $g$ be any primitive root modulo $p$, and $\{a_1, \ldots,
a_{\phi(p-1)}\}$ be a list of all positive integers smaller than $p-1$
and relatively prime to $p-1$. It follows that $g^{a_i}$ are all
distinct primitive roots modulo $p$. Their product is precisely
$g^{\sum a_i}$. We will therefore investigate $\sum a_i$.

Note that if $k$ is relatively prime to $p-1$, so is $p-1-k$. $k \ne
p-1-k$, unless $2k = p-1$, but the only way a number $k$ such that $2k
= p-1$ is relatively prime to $p-1$ is when $k = 1$, $p-1 = 2$, and $p
= 3$. We treat this case separately. Assume therefore that $p \ne 3$.
We can divide numbers less than $p-1$ and relatively prime to $p-1$
into $\phi(p-1)/2$ pairs, the sum of numbers in each pair is equal to
$p-1$. Therefore, the sum of all numbers smaller than $p-1$ and
relatively prime to $p-1$ is exactly $(p-1)\phi(p-1)/2$.

Now, $g^{(p-1)\phi(p-1)/2} = (g^{(p-1)/2})^{\phi(p-1)}$. We have
$g^{(p-1)/2} \equiv -1 \pmod p$, and so the desired congruence
follows.

For $p=3$, there's only one primitve root modulo $3$, which is $2 =
\equiv -1 \pmod 3$. On the other hand $(-1)^{\phi(p-1)} =
(-1)^{\phi(2)} = -1 \pmod 3$.

\paragraph{Ex. 4.10}
Show that the sum of all the primitive roots modulo $p$ is
congruent to $\mu(p-1)$ modulo $p$.

For any divisor $k$ of $p-1$, let $\psi(k)$ denote the sum of elements
of order $k$ modulo $p$. We'll show that $\psi(k) = \mu(k)$, which
implies the desired congruence.

First note that $\psi$ is multiplicative modulo $p$.

If $m, n$ are relatively prime, then $\Z/(mn) \simeq \Z/(m) \times
\Z/(n)$ through isomorphism $a \mapsto (a \mod m, a \mod n)$ (Chinese
remainder theorem).  From this isomorphism it follows that every $c$
of the order $mn$ can be uniquely written as a product $c = ab$, with
$a$ having order $m$, and $b$ order $n$.

Let $a_1, \ldots, a_s$ be all elements of $U(\Z/p\Z)$ of order $m$,
and $b_1, \ldots, b_t$ be all elements of $U(\Z/p\Z)$ of order
$n$. Since $a_i$ and $b_j$ are all elements of a cyclic subgroup
$C_{mn}$ of $U(\Z/p\Z)$ of elements of orders dividing $mn$, the above
reasoning applies. Therefore:

\begin{equation}
  \psi(m)\psi(n) = (a_1 + \ldots + a_s)(b_1 + \ldots + b_t) = \sum_{1
    \leq i \leq s} \sum_{1 \leq j \leq t} a_i b_j \equiv \psi(mn) \pmod p
\end{equation}

We now show that $\psi(p^n) \equiv \mu(p^n) \pmod p$, which along with
multiplicativity gives us desired conclusion.

We proceed by induction. For $n = 0$, the only element of order $p^0 =
1$ is $1$, and its sum is $1 = \mu(1)$.

For $n > 0$, the sum of all elements of order $p^n$ is the sum of all
roots of $x^{p^n} \equiv 1 \pmod p$ minus the roots of $x^{p^{n-1}}
\equiv 1 \pmod p$. Both of these sums are equal to $0$ except in case
when $p^{n-1} = 1$, that is, $n = 1$. In that case, there is only one
root, $1$, so $\psi(p) = -1$, and $\psi(p^n) = 0$ for $n>1$.


\paragraph{Ex. 4.11}
Prove that $1^k + 2^k + . . . + (p-1)^k \equiv 0 \pmod p$ if $p-1 \not
| k$, and $\equiv -1 \pmod p$ if $p-1 | k$.

If $g$ is a primitive root, then $g, g^2, \ldots, g^{p-1}$ is a
permutation of $1, 2, \ldots, p-1$ modulo $p$. Therefore:

\begin{equation}
  1^k + 2^k + . . . + (p-1)^k \equiv g^k + (g^2)^k + \ldots +
  (g^{p-1})^k \equiv g^k + (g^k)^2 + \ldots + (g^k)^{p-1} \pmod p
\end{equation}

If $p-1|k$, $x^k = 1$, and so $1 + 1^2 + \ldots + 1^{p-1} = p-1 \equiv
-1 \pmod p$. On the other hand, if $p-1 \not | k$, then $g^k \ne
1$. Write $x = g^k$. We want to show that $x + x^2 + \ldots + x^{p-1}
\equiv 0 \pmod p$. We have:

\begin{equation}
x + x^2 + \ldots + x^{p-1} = x(1 + x + \ldots + x^{p-2}) \equiv
x(x^{p-1} - 1)(x-1)^{-1} \equiv 0 \pmod p
\end{equation}

since $x^{p-1} - 1 \equiv 0 \pmod p$.

\paragraph{Ex. 4.12}
Use the existence of a primitive root to give another proof of Wilson's theorem
$(p - 1)! \equiv -1  \pmod p$.

If $g$ is a primitive root, then $g, g^2, \ldots, g^{p-1}$ is a
permutation of $1, 2, \ldots, p-1$ modulo $p$. Therefore:

\begin{equation}
  (p-1)! \equiv g^1 \cdot g^{p-1} \equiv g^{1 + 2 + \ldots + p-1}
  \equiv g^{p(p-1)/2} \equiv (g^p)^{(p-1)/2} \equiv g^{(p-1)/2} \equiv
  -1 \pmod p
\end{equation}


\paragraph{Ex. 4.13}
Let $G$ be a finite cyclic group and $g \in G$ a generator. Show that
all the other generators are of the form $g^k$, where $(k, n) = 1$,
$n$ being the order of $G$.

If $(k, n) = d \ne 1$, then $n | \frac{kn}{d}$, and so $(g^k)^{n/d} = g^{kn/d}
= 1$. Thus the order of $g^k$ divides $n/d$, which implies that it's
smaller than $n$, and so $g^k$ cannot generate $G$.

On the other hand, if $(k, n) = 1$, let $ak + bn = 1$. Then $g =
g^{ak+bn} = g^{ak}g^{bn} = (g^k)^a (g^n)^b = (g^k)^a 1^b =
(g^k)^a$. Thus $g^k$ generates $g$, and so generates the whole group.


\paragraph{Ex. 4.14}
Let $A$ be a finite abelian group and $a, b \in A$ elements of order
$m$ and $n$, respectively.  If $(m, n) = 1$, prove that $ab$ has order
$mn$.

Let $M$ and $N$ be subgroups generated by $m$ and $n$
respectively. Consider $a \in M \cap N$. Its order $d$ must divide the
order of $M$ and the order of $N$, but since $(m, n) = 1$, we
necessarily have that $d = 1$, and so $M \cap N = \{1\}$.

Note that $(ab)^{mn} = (a^m)^n (b^n)^m = 1$, so the order of $ab$ must
divide $mn$. Now, let $e$ be the order of $ab$, that is, $(ab)^e =
1$. Then $a^e b^e = 1$. and so $a^e = b^{-e}$. It follows that $a^e
\in B$, and $b^e \in A$, so $a^e = b^e = 1$. Thus, $m|e$ and $n|e$,
and so $e$ is a multiple of both $m$ and $n$. Since $(m, n) = 1$,
their smallest multiple is $mn$, and since we know that the order must
divide $mn$, we get that the order is indeed $mn$.

\paragraph{Ex. 4.15}
Let $K$ be a field and $G \subset K^*$ a finite subgroup of the
multiplicative group of $K$.  Extend the arguments used in the proof
of Theorem 1 to show that $G$ is cyclic.

Let $n = |G|$. If $K$ has characteristic $0$, consider an extension
$\Q \subset \Q(G)$. We say this is an algebraic extension: indeed,
every element of $G$ is a root of polynomial $x^{n} - 1$. Thus we can
embed $\Q(G)$ in $\C$. In this embedding, $G$ is a subgroup of the
group of roots of $x^{n} - 1$. But in $\C$, this group has exactly $n$
elements, and is cyclic, generated by $e^{2i \pi / n}$. Thus, $G$ is a
subgroup of a cyclic group, and so is cyclic itself.

If $K$ has characteristic $p$, consider an extension
$\mathbb{F}_p(G)$. Arguing as before, we say that this is an algebraic
extension, but more importantly, it is also a finite extension, so
$\mathbb{F}_p(G)$ is a finite field with $p^k$ elements, for some
$k$. In this embedding, $G$ is a subgroup of the multiplicative group
of $\mathbb{F}_{p^k}$. We now need to show that this group is
cyclic.

The polynomial $x^{p^k - 1} - 1$ has exactly $p^k - 1$ roots in
$\mathbb{F}_{p^k}$. We now argue just like in proof of the Theorem 1
-- if $d|p^k - 1$, then $x^d - 1 = 0$ has exactly $d$ solutions in
$\F_{p^k}$. Indeed, just like in Proposition 4.1.2, we prove that we
have:
\begin{equation}
  x^{p^k - 1} - 1 = (x^d - 1)g(x)
\end{equation}
for some (easy to calculate) polynomial $g(x)$. If $x^d - 1$ had less
than $d$ roots, then $x^{p^k - 1} - 1$ necessarily would have less
than $p^k - 1$ roots as well, but it contradicts the fact that all
nonzero elements of $\F_{p^k}$ are roots of that polynomial, and there
are $p^k - 1$ of them.

So, we continue just like in the proof of Theorem 1: we let $\psi(d)$
equal number of elements of $U(\F_{p^k})$ of order $d$, using
reasoning above we conclude that $d = \sum_{c|d} \psi{c}$, we use
Moebius inversion to conclude that $\psi(d) = \phi(d)$, and note that
$\phi(p^k - 1) > 1$, so there exists an element of order $p^k - 1$.

\paragraph{Ex. 4.16}
Calculate the solutions to $x^3 \equiv 1 \pmod {19}$ and $x^4 \equiv 1
\pmod {17}$.

Note that $x^3 - 1 = (x-1)(x^2 + x + 1)$. We complete the square in
$x^2 + x + 1$: the discriminant is $1 - 4 = -3 \equiv 16 \pmod {19}$,
and so its square root is $4 \pmod {19}$. The inverse of $2$ modulo
$19$ is $10$, as $2 \cdot 10 = 20 \equiv 1 \pmod {19}$. Thus:
\begin{equation}
x^2 + x + 1 = (x - 2^{-1}\cdot(-1 - 4))(x-2^{-1}\cdot(-1+4)) = (x-10 \cdot
14)(x-10 \cdot 3) = (x - 7)(x - 11)
\end{equation}

We deal with the second equation in a similar manner: $x^4 - 1 = (x^2
- 1)(x^2 + 1) = (x-1)(x+1)(x^2+1)$, so it remains to find square roots
of $-1 \equiv 16 \pmod {17}$, but these are just $4$ and $-4 \equiv 13
\pmod {17}$.

\paragraph{Ex. 4.17}
Use the fact that $2$ is a primitive root modulo $29$ to find the
seven solutions to $x^7 \equiv 1 \pmod {29}$.

We can write $x \equiv 2^k \pmod {29}$ for some $k \in \{0, 1, \ldots,
27\}$. For $(2^k)^7 \equiv 1 \pmod {29}$ to be true, $7k$ must be a
multiple of $28$. This means that $k \in \{0, 4, 8, 12, 16, 20, 24\}$,
and corresponding values of $x$ are $1, 16, 24, 7, 25, 23, 20 \pmod
{29}$.

\paragraph{Ex. 4.18}
Solve the congruence $1 + x + \ldots + x^6 \equiv 0 \pmod {29}$.

Note that $(1-x)(1 + x + \ldots + x^6) = 1 - x^7$. From the previous
exercise we can find the roots to the right hand side, and except of
$x = 1$, the remainding $6$ solutions must be the solutions to $1 + x
+ \ldots + x^6 \pmod {29}$.

\paragraph{Ex. 4.19}
Determine the numbers $a$ such that $x^3 \equiv a \pmod {p}$ is
solvable for $p = 7, 11, 13$.

This tedious exercise can be easily solved by a brute-force search, so
I'll skip it.

\paragraph{Ex. 4.20}
Let $p$ be a prime, and $d$ a divisor of $p - 1$. Show that $d$th
powers form a subgroup of $U(\F_p)$ of order $(p-1)/d$. Calculate this
subgroup for $p = 11, d = 5$, for $p = 17, d = 4$, and for $p = 19, d
= 6$.

Let $a = x^d, b = y^d$. Then obviously $ab = (xy)^d$, so $ab$ is $d$th
power. Similarly, $a^{-1} = (x^{-1})^d$, so $a^{-1}$ is also $d$th
power. Thus we see that $d$th powers form a subgroup.

If $x = y^d$ for some $y$, then $x^{(p-1)/d} - 1 = (y^d)^{(p-1)/d} -1
= y^{p-1} - 1 = 1 - 1 = 0$, so there are at most $(p-1)/d$ $d$th
powers. Consider a function $f: U(\F_p) \to U(\F_p), f(a) = a^d$. The
image has at most $(p-1)/d$ elements. Each element $b$ in the image
has at most $d$ elements in the preimage, as they all must be
solutions to $x^d = b$. If the image had strictly less than $(p-1)/d$
elements, the preimage of the image, which is $U(\F_p)$, would need to
have strictly less than $d \cdot (p-1)/d = p-1$ elements, which is a
contradiction.

For $p = 11$, the subgroup of 5th powers have only $(11-10)/5 = 2$
elements. One of them is obviously 1, the other is e.g. $2^5 = 32
\equiv -1 = 10 \pmod {11}$.

For $p = 17$, the group of $4$th elements has $(17-1)/4 = 4$
elements. These are $1$, $2^4 = 16 \equiv -1 \pmod {17}$, $3^4 = 81
\equiv 13 \pmod {17}$, and $-13 \equiv 4 \pmod {17}$.

For $p = 19$, the subgroup of $6$th power has $3$ elements, which are
$1$, $2^6 = 64 \equiv 7 \pmod {19}$, and $7^2 = 49 \equiv 11 \pmod
{19}$.

\paragraph{Ex. 4.21}
If $g$ is a primitive root modulo $p$, and $d|p-1$, show that
$g^{(p-1)/d}$ has order $d$. Show also that $a$ is a $d$th power iff
$a \equiv g^{kd} \pmod p$ for some $k$. Do Exercises 4.16-4.20 making
use those observations.

These observations are trivial, and we already used them above.

\paragraph{Ex. 4.22}
If $a$ has order $3$ modulo $p$, show that $1+a$ has order $6$.

We need to show that $(1+a)^k$ is not equal to $1$ for $k < 6$, and
that $(1+a)^6 = 1$.

Note that as $a$ has order $3$ modulo $p$, we have:
\begin{equation}
  0 = a^3 - 1 = (a-1)(1+a+a^2)
\end{equation}
As $a$ has order $3$, $a \ne 1$, so $1+a+a^2 = 0$.

Now, let us handle all cases separately.

$k=1$: Since $a$ has order 3, $a \ne 0$, and so $(1+a)^1 \ne 1$.

$k=2$: We have $(1+a)^2 = 1 + 2a + a^2 = 1 + a + a^2 + a = a$. We know
that $a \ne 1$.

$k=3$: We have $(1+a)^3 = 1 + 3a + 3a^2 + a^3 = 3(1+a+a^2)-1 = -1$. We
have $-1 \ne 1$, otherwise $2 = 0$, and $a$ cannot have order $3$.

$k=4$: We have $(1+a)^4 = 1 + 4a + 6a^2 + 4a^3 + a^4 = 1 + 4a + 6a^2 +
4 + a = 5(1 + a + a^2) + a^2 = a^2$, which is not equal to $1$, as $a$
would have been order $2$ otherwise.

$k=5$: We have $(1+a)^5 = 1 + 5a + 10a^2 + 10a^3 + 5a^4 + a^5 = 1 + 5a
+ 10a^2 + 10 + 5a + a^2 = 11 + 10a + 11a^2 = 11(1+a+a^2) - a = -a$,
and $-a \ne 1$, as otherwise $a =-1$, which doesn't have order $3$.

$k=6$: We have $(1+a)^6 = 1 + 6a + 15a^2 + 20a^3 + 15a^4 + 6a^5 + a^6
= 1 + 6a + 15a^2 + 20 + 15a + 6a^2 + 1 = 22 + 21a + 21a^2 = 1 +
21(1+a+a^2) = 1$.

\paragraph{Ex. 4.23}
Show that $x^2 \equiv -1 \pmod p$ has a solution iff $p \equiv 1 \pmod
4$, and that $x^4 \equiv -1 \pmod p$ has a solution iff $p \equiv 1
\pmod 8$.

For the first congruence, we use Proposition 4.2.1 to conclude that
existence of 2nd power residue of $-1$ modulo prime $p$ is equivalent
to $(-1)^{(p-1)/2} \equiv 1 \pmod p$, which in turn is equivalent to
$(p-1)/2$ being even, that is, $p \equiv 1 \pmod 4$.

We do the same for the second congruence: $-1$ is a 4th power residue
modulo $p$ if $(-1)^{(p-1)/d} \equiv 1 \pmod p$, where $d = (4,
p-1)$. Note that if $a$ is 4th power residue, then it is 2nd power
reside too, so necessarily $p \equiv 1 \pmod 4$, and so $d =
4$. Therefore, we are asking when $(-1)^{(p-1)/4} \equiv 1 \pmod p$,
which in turn is equivalent to $(p-1)/4$ being even, which happens
whenever $p \equiv 1 \pmod 8$.

\paragraph{Ex. 4.24}
Show that $a x^m + b y^n \equiv c \pmod p$ has the same number of
solutions as $a x^{m'} + b y^{n'} \equiv c \pmod p$, where $m' =
(m,p-1)$ and $n' = (n, p-1)$.

%% Every solution to the first equation yields the solution to the
%% second: indeed, $m = m' u$, $n = n' v$, and we can write $a x^m + b
%% y^n = a (x^u)^{m'} + b (y^v)^{n'}$. 

%% Now consider a solution $(x, y)$ to $a x^{m'} + b y^{n'} \equiv c
%% \pmod p$. Let $g$ be a primitive root modulo $p$, and write $x = g^e$,
%% $y = g^f$.

\section{Chapter 5}

\paragraph{Ex. 5.1}
Use Gauss' lemma to determine $\left(\frac{5}{7}\right),
\left(\frac{3}{11}\right), \left(\frac{6}{13}\right),
\left(\frac{-1}{p}\right)$.

$\left(\frac{5}{7}\right)$: $(7-1)/2 = 3$, and so least residues of $1
\cdot 5, 2 \cdot 5, 3 \cdot 5$ are $-2, 3, 1$ respectively, so $\mu =
1$ and $\left(\frac{5}{7}\right) = (-1)^\mu = -1$.

$\left(\frac{3}{11}\right)$: $(11-1)/2 = 5$, and so least residues of
$1 \cdot 3, 2 \cdot 3, 3 \cdot 3, 4 \cdot 3, 5 \cdot 3$ are $3, -1,
-2, 1, 4$ respectively, so $\mu = 2$ and $\left(\frac{3}{11}\right) =
(-1)^\mu = 1$.

$\left(\frac{6}{13}\right)$: $(13-1)/2 = 6$, the least residues of
first $6$ multiples of $6$ are $1, -1, 6, -2, 4, -3$, $\mu = 3$,
$\left(\frac{6}{13}\right) = (-1)^\mu = -1$.

$\left(\frac{-1}{p}\right)$: the least residues of the first $(p-1)/2$
multiples of $-1$ are all negative, so $\left(\frac{-1}{p}\right) =
(-1)^\mu = (-1)^{(p-1)/2}$.

\paragraph{Ex. 5.2}
Show that the number of solutions to $x^2 \equiv a \pmod p$ is equal
to $1 + (a/p)$.

If $x$ is a solution to the given congruence, $-x$ is also a solution,
and it's a different solution unless $x \equiv 0$ or $x \equiv -x
\pmod p$, but $p$ is odd, so it cannot happen. Since the equation is
of degree 2 over a field, there cannot be any more solutions.

If $a \equiv 0 \pmod p$, there's only one solution, $x \equiv 0 \pmod
p$, but also $1+(0/p) = 1+0 = 1$.
 
Otherwise, there are either 0 or 2 solutions. There are 0 solutions
whenever $a$ is not a quadratic residue, that is, $(a/p) = -1$, and $1
+ (a/p) = 1 + (-1) = 0$ in that case. In the other case, $(a/p) = 1$,
and so $1 + (a/p) = 1 + 1 = 2$. In all cases the expression $1 +
(a/p)$ equals the number of solutions.

\paragraph{Ex. 5.3}
Suppose $p \not | \; a$. Show that the number of solutions to $a x^2 +
bx + c \equiv 0 \pmod p$ is equal to $1 + (b^ - 4ac/p)$.

High school algebra gives tells us that the solutions correspond to
quadratic residues of $\Delta = b^2 - 4ac$, and so we can just use the
result from the previous exercise.

\paragraph{Ex. 5.4}
Prove that $\sum_{a=1}^{p-1} (a/p) = 0$.

This is clear: there are exactly as many quadratic residues as there
are nonresidues, and for residues $(a/p) = 1$, and for nonresidues it
equals $-1$.

\paragraph{Ex. 5.5}
Prove that $\sum_{x=1}^{p-1} ((ax + b)/p) = 0$ provided that $p \not | \; a$.

If $p \not | \; a$, then $\{a \cdot 1, a\cdot 2, \ldots, a \cdot (p-1)\}$ is a
complete set of residues modulo $p$, and thus also is $\{a \cdot 1+b, a\cdot 2
+b, \ldots, a \cdot (p-1) +b\}$, so we just use the result of the
previous exercise.

\paragraph{Ex. 5.6}
Show that the number of solutions to $x^2 - y^2 \equiv a \pmod p$ is given by:

\begin{equation}
  \sum_{y=0}^{p-1}(1+((y^2 + a)/p))
\end{equation}

This is pretty clear: the equation is equivalent to $x^2 \equiv y^2 +
a \pmod p$, which, if we hold $y$ fixed, by Ex. 5.2 has $(1+((y^2 +
a)/p))$, so if we let $y$ vary from $1$ to $p-1$ and sum the number of
solutions for each $y$, we get the result.

\paragraph{Ex. 5.7}
By calculating directly show that the number of solutions to $x^2 -
y^2 \equiv a \pmod p$ is $p-1$ if $p \not | \; a$, and $2p - 1$ if $p
| a$.

We change variables $u = x+y$, $v = x-y$. The equation then takes form
$uv \equiv a \pmod p$.

If $p | a$, that is, $a \equiv 0 \pmod p$, then there are exactly $2p -
1$ solutions corresponding to $u = 0$ and any $v$, to $v = 0$ and any
$u$. Each of these gives us $p$ solutions, but we get $u = v = 0$ for
both, so there are $2p - 1$ solutions overall. All of these are
obtainable from original variables: we obtain $u = 0$ and any $v$ by
letting $x = -y = v \cdot 2^{-1}$, and $v = 0$ and any $u$ by letting
$x = y = u \cdot 2^{-1}$.

Otherwise, if $p \not | \; a$, as we fix $v$ and vary $u$ from $1$ to
$p-1$, the values of $uv$ are all nonzero residues mod $p$. Exactly
one of them is congruent to $a$. We can fix $v$ to $p-1$ different
values, and so we get $p-1$ different solutions. Note also that our
variable substitution is bijective.

\paragraph{Ex. 5.8}
Combining the results of Ex. 5.6 and 5.7 show that:

\begin{equation}
  \sum_{y=0}^{p-1} \left(\frac{y^2 +a}{p}\right) =
  \begin{cases}
    -1 \quad \textrm{if}\, p\not | \; a \\
    p-1 \quad \textrm{if}\, p | a
  \end{cases}
\end{equation}

Immediate.

\paragraph{Ex. 5.9}
Prove that $1^2 3^2 \cdots (p-2)^2 \equiv (-1)^{(p-1)/2} \pmod p$ using Wilson's theorem.

Note that $a^2 \equiv a(a-p) \equiv -a(p-a) \pmod p$, and so:
\begin{equation}
  1^2 3^2 \cdots (p-2)^2 \equiv (-1)^{(p-1)/2}
  1(p-1)3(p-3)\cdots(p-2)2 \equiv (-1)^{(p-1)/2} (-1) \equiv
  (-1)^{(p+1)/2} \pmod p
\end{equation}

\paragraph{Ex. 5.10}
Let $r_1, r_2, \ldots, r_{(p-1)/2}$ be the quadratic residues between
$1$ and $p$. Show that their product is congruent to $1$ modulo if $p
\equiv 3 \pmod 4$, and to $-1$ if $p \equiv 1 \pmod 4$.

This is the same problem as Ex 5.9 -- $1^2, 3^2, \ldots, (p-2)^2$ are
all quadratic residues, as each quadratic residue is a square of two
different residues, namely $a$ and $p-a$, at least one of which occurs
on the list above.

\paragraph{Ex. 5.11}
Suppose that $p \equiv 3 \pmod 4$, and that $q = 2p+1$ is also
prime. Prove that $2^p -1$ is not prime.

We'll show that $2^p \equiv 1 \pmod q$. Note that $p = (q-1)/2$, and
so $2^p = 2^{(q-1)/2} \equiv (2/q) \pmod q$. Consider then $(2/q)$. It
is equal to $(-1)^{(q^2 - 1)/8}$. The exponent is equal to $(4p^2 +
4p)/8 = (p^2 + p)/2$. Since $p \equiv 3 \pmod 4$, $p^2 + p \equiv 0
\pmod 4$, and so $(p^2 + p)/2$ is even. Therefore $(2/q) = 1$, and so
$2^p \equiv (2/q) = 1 \pmod q$.

\paragraph{Ex. 5.12}
Let $f(x) \in \Z[x]$. We say that a prime $p$ divides $f(x)$ if
there's an integer $n$ such that $p|f(n)$. Describe the prime divisors
of $x^2 + 1$ and $x^2 -2$.

We are asking for which $p$, $x^2 + 1 \equiv 0 \pmod p$, and for which
$p$, $x^2 - 2 \equiv 0 \pmod p$. This is equivalent to asking for
which $p$, $-1$ and $2$ are quadratic residue. Quadratic reciprocity
answers these questions in a satisfactory way.

\paragraph{Ex. 5.13}
Show that any prime divisor of $x^4 - x^2 + 1$ is congruent to $1$
modulo $12$.

Suppose that $x$ is such that $x^4 - x^2 + 1 \equiv 0 \pmod p$. We use
Ex. 5.3 to conclude that $1^2 - 4 \cdot 1 \cdot 1 = - 3$ is a
quadratic residue modulo $p$, that is, $(-3/p) = 1$. We have:

\begin{equation}
  \label{eq:3p-legendre}
  1 = \left(\frac{-3}{p}\right) =
  \left(\frac{-1}{p}\right)\left(\frac{3}{p}\right) = (-1)^{(p-1)/2}
  \left(\frac{3}{p}\right)
\end{equation}

On the other hand, the quadratic reciprocity tells us that:
\begin{equation}
  \label{eq:3p-quadr-recipr}
  \left(\frac{3}{p}\right) \left(\frac{p}{3}\right) = (-1)^{(3-1)/2 \cdot (p-1)/2} = (-1)^{(p-1)/2}
\end{equation}

Combining (\ref{eq:3p-legendre}) with (\ref{eq:3p-quadr-recipr}), we get:

\begin{equation}
  \left(\frac{p}{3}\right) = 1
\end{equation}

This implies that $p \equiv 1 \pmod 3$. It remains to prove that $p
\equiv 1 \pmod 4$, which is equivalent to showing that $-1$ is a
quadratic residue modulo $p$.

Write the original equation in the form $x^2(x^2 - 1) \equiv -1 \pmod
p$. As $x^2$ is obviously a quadratic residue, if we show that $x^2 -
1$ is also a quadratic residue, then $-1$ will also be a quadratic
residue, as a product of two residues.

But again we rewrite the original equation as $x^4 \equiv x^2 - 1
\pmod p$, from which it is clear that $x^2 - 1$ is a quadratic
(even biquadratic) residue.

\paragraph{Ex. 5.14}
Use the fact that $U(\Z/p\Z)$ is cyclic to give a direct proof that
$(-3/p) = 1$ when $p \equiv 1 \pmod 3$.

Since $p \equiv 1 \pmod 3$, $3$ divides $p-1$, and so by Cauchy
theorem, there's an element $\rho \in U(\Z/p\Z)$ of order $3$. We'll
show that $(2 \rho + 1)^2 \equiv -3 \pmod p$.

Since $\rho^3 \equiv 1$, $1 + \rho + \rho^2 \equiv 0$. Then $4 \rho^2
+ 4 \rho + 4 \equiv 0$, so $(2 \rho + 1)^2 \equiv -3$.

\paragraph{Ex. 5.15}
If $p \equiv 1 \pmod 5$, show directly that $(5/p) = 1$ by the method
of Ex. 5.14.

Since $p \equiv 1 \pmod 5$, $5$ divides $p-1$, and so by Cauchy
theorem, there's an element $\rho \in U(\Z/p\Z)$ of order $5$.

Since $\rho$ has order $5$, $1 + \rho + \rho^2 + \rho^3 + \rho^4 =
0$. Then, simple calculation shows that $(\rho + \rho^4)^2 + (\rho +
\rho^4) - 1 = 0$. This means that $x^2 + x - 1 \equiv 0 \pmod p$ is
solvable, which by Ex. 4.3 is equivalent to $1^2 - 4\cdot1\cdot(-1) =
5$ being a quadratic residue.

\paragraph{Ex. 5.16}
Using quadratic reciprocity find the primes for which $7$ is quadratic
residue. Do the same for $15$.

Quadratic reciprocity tells us that:

\begin{equation}
  \left(\frac{p}{7}\right)\left(\frac{7}{p}\right) = (-1)^{(p-1)/2
    \cdot (7-1)/2} = (-1)^{3(p-1)/2} = (-1)^{(p-1)/2}
\end{equation}

Let us now consider $(p/7)$. The quadratic residues modulo $7$ are $1,
2, 4$. Thus, $(7/p) = 1$ whenever $p$ is congruent to either of $1, 2,
4$ modulo $7$ and $1$ modulo $4$, or to $3, 5, 6$ modulo $7$ and $3$
modulo $4$. We can sum up these conditions to: $p$ must be congruent
to either of $1, 2, 4, 9, 15, 18$ modulo $28$.

The process for $15$ is mostly the same, save for a minor details that
$15$ is actually not a prime. However, we note that $(15/p) =
(3/p)(5/p)$, and then we proceed to determine the quadratic character
of $3$ and $5$.

\paragraph{Ex. 5.17}
Supply the details to the proof of Proposition 5.2.1 and to the corollary to the lemma
following it.

This is too trivial to bother writing it down.

\paragraph{Ex. 5.18}
Let $D$ be a square-free integer that is also odd and positive. Show
that there's an integer $b$ prime to $D$ such that $(b/D) = -1$.

Let $D = p_1 \cdots p_k$. By definition:
\begin{equation}
  \label{eq:gener-legendre}
  \left(\frac{b}{D}\right) = \left(\frac{b}{p_1}\right) \cdots \left(\frac{b}{p_k}\right)
\end{equation}

Let $b_1$ be any quadratic non-residue modulo $p_1$, and $b_i$ for $i
> 1$ be any quadratic residues modulo $p_i$. By Chinese remainder
theorem, there exists $b$ such that $b \equiv b_i \pmod {p_i}$, and by
(\ref{eq:gener-legendre}), it is clear that $(b/D) = -1$.

\paragraph{Ex. 5.19}
Let $D$ be as in Exercise 18. Show that $\sum (a/D) = 0$, where the
sum is over a reduced residue system modulo $D$. Conclude that exactly
one half of the elements in $U(\Z/D\Z)$ satisfy $(a/D) = 1$.

If $D$ is prime itself, then the proposition follows from the
corresponding property of the regular Legendre symbol (there are
exactly as many residues as there are nonresidues). Therefore, let $D
= p_1 \ldots p_k$, $p_i$ be all different, and $k > 1$. Let $D' = p_2
\ldots p_k$, so that $D = p_1 D'$. Then:

\begin{equation}
  \sum_{\mod D} \left(\frac{a}{D}\right) = \sum_{\mod D}
  \left(\frac{a}{p_1}\right) \left(\frac{a}{D'}\right)
\end{equation}
 
where $\sum_{\mod D}$ denotes sum over reduced residues modulo $D$.

By Chinese remainder theorem, each reduced residue $a$ modulo $D$
corresponds to a pair of (residue modulo $p_1$, residue modulo $D'$)
through coordinate-wise reduction. All residues modulo $D$ arise this
way. Thus, we can write:

\begin{equation}
  \sum_{\mod D} \left(\frac{a}{p_1}\right) \left(\frac{a}{D'}\right) =
  \left(\sum_{\mod p_1} \left(\frac{a}{p_1}\right)\right)
  \left(\sum_{\mod D'}\left(\frac{a}{D'}\right)\right)
\end{equation}

But now again $\sum_{\mod p_1} \left(\frac{a}{p_1}\right) = 0$ by
properties of regular Legendre symbol, so the whole sum must be $0$.

Since $\sum_{\mod D} \left(\frac{a}{D}\right) = 0$, and each summand
is either $1$ or $-1$, there must be exactly as many $1$s as there are
$-1$s.

\paragraph{Ex. 5.20}
Let $a_1, a_2, \ldots, a_{\phi(D)/2}$ be integers between $1$ and $D$
such that $(a_i, D) = 1$ and $(a_i/D) = 1$. Prove that $D$ is a
quadratic residue modulo a prime $p \not | \; D$, $p \equiv 1 \pmod 4$
iff $p \equiv a_i \pmod D$ for some $i$.

We are interested in calculating $(D/p)$. Generalization of quadratic
reciprocity to Jacobi symbol gives us the following relation:

\begin{equation}
  \left(\frac{p}{D}\right)\left(\frac{D}{p}\right) = (-1)^{\frac{p-1}{2}\frac{D-1}{2}} = 1
\end{equation}

Thus $(D/p) = (p/D)$. But $(p/D) = 1$ exactly if $p \equiv a_i \pmod D$ for some $i$.

\paragraph{Ex. 5.21}
Apply the method of Ex. 5.19 and 5.20 to find those primes for which
$21$ is a quadratic residue.

Let $D = 21 = 3 \cdot 7$, $\phi(21) = (3-1)(7-1) = 2 \cdot 6 =
12$. Thus there are $\phi(21)/2 = 6$ residues $a$ modulo $21$ such
that $(a/21) = 1$. These are either residues, which are quadratic
residues of both $3$ and $7$, or residues which are quadratic
nonresidues of both $3$ and $7$. There's a single quadratic residue
mod $3$, which is $1$, and a single nonresidue, which is $2$. There
are $3$ quadratic residues mod $7$, which are $1$, $2$, and $4$, and
$3$ nonresidues, which are $3$, $5$ and $6$. Thus, if $a$ is such that
$(a/21) = 1$, then these are the possibilities for values of $(a \mod
3, a \mod 7)$: $(1, 1), (1, 2), (1, 4), (2, 3), (2, 5), (2,6)$. We can
recover residues mod $21$ from these pairs using Chinese remainder
theorem. Then, the solution to exercise amounts to applying the previous exercise.

\paragraph{Ex. 5.22}
Use the Jacobi symbol to determine $(113/997)$, $(215/761)$,
$(514/1093)$, and $(401/757)$.

We'll do only the first one, $(113/997)$. We have:

\begin{equation}
  \left(\frac{113}{997}\right)\left(\frac{997}{113}\right) =
  (-1)^{\frac{113-1}{2}\frac{997-1}{2}} = (-1)^{56 \cdot 498} = 1
\end{equation}

Thus $(113/997) = (997/113) = (93/113)$. We apply reciprocity again:

\begin{equation}
  \left(\frac{93}{113}\right)\left(\frac{113}{93}\right) =
  (-1)^{\frac{93-1}{2}\frac{113-1}{2}} = (-1)^{46 \cdot 56} = 1
\end{equation}

Thus $(93/113) = (113/93) = (20/93) = (4/93)\cdot(5/93) = (5/93)$, as
$4$ is always a quadratic residue. Thus we have:

\begin{equation}
  \left(\frac{5}{93}\right)\left(\frac{93}{5}\right) =
  (-1)^{\frac{93-1}{2}\frac{5-1}{2}} = (-1)^{46 \cdot 2} = 1
\end{equation}

And so $(5/93) = (93/5) = (3/5) = -1$, as $1, 4$ are the only
quadratic residues modulo $5$.

\paragraph{Ex. 5.23}
Suppose that $p \equiv 1 \pmod 4$. Show that there exist integers $s$
and $t$ such that $pt = 1 + s^2$.  Conclude that $p$ is not a prime in
$\Z[i]$. Remember that $\Z[i]$ has unique factorization.

Since $p \equiv 1 \pmod 4$, $(-1)^{(p-1)/2} = 1$, and so $-1$ is a
quadratic residue. Thus there exists $s \leq p-1$ such that $s^2
\equiv -1 \pmod p$, which is the same as saying that $p$ divides $s^2
+ 1$, that is, $pt = s^2 + 1$. 

In $\Z[i]$ we can factor $pt = s^2 + 1$ as $pt = (s-i)(s+i)$. If $p$
was a prime, it would have to divide one of $s+i, s-i$, and so $p^2 =
||p|| \leq \max(||s-i||, ||s+i||) = \max(s^2 + 1, s^2 + 1) = s^2 + 1
\leq p^2 - 2p + 1 + 1$, and so $2p \leq 2$, which is a contradiction.

\paragraph{Ex. 5.24}
If $p \equiv 1 \pmod 4$, show that $p$ is a sum of two squares,
i.e. $p = a^2 + b^2$ with $a, b \in \Z$.

Since $p$ is a nonprime in $\Z[i]$, we can write $p = \alpha \beta$
for some nonunit $\alpha, \beta \in \Z[i]$. We then have:

\begin{equation}
  p^2 = ||p|| = ||\alpha \beta|| = ||\alpha|||\cdot||\beta|| = (x^2 + y^2)(c^2 + d^2)
\end{equation}

Since $p$ is prime, both sides of above equality are integers, and
neither $\alpha$ nor $\beta$ are units, we must have $p = x^2 + y^2 =
c^2 + d^2$. Now it only remains to show that neither of, say, $x$ or
$y$, is $0$. If $y = 0$, then $\alpha \in \Z$, and so $\beta \in \Z$,
cause otherwise if $\beta \not \in \Z$, $p = \alpha \beta \not \in
\Z$, which is a contradiction. But if $\alpha, \beta \in \Z$, they
would give a nontrivial factorization of $p$ in $\Z$, which cannot
exist. We deal with the case of $x = 0$ by similar argument.

\paragraph{Ex. 5.25}
An integer is called a biquadratic residue modulo $p$ if it is
congruent to a fourth power. Using the identity $x^4 + 4 = ((x + 1)^2
+ 1)((x - 1)^2 + 1)$ show that $-4$ is a biquadratic residue modulo $p$
iff $p \equiv 1 \pmod 4$.

In one direction it's trivial: if $-1$ is a biquadratic residue, it
necessarily is also a quadratic residue, and so $p \equiv 1 \pmod
4$. In the other direction, assume that $p \equiv 1 \pmod 4$, so that
$-1$ is a quadratic residue. Thus there's an $x$ such that $(x+1)^2
\equiv -1 \pmod p$, and so $x^2 + 4 \equiv ((x+1)^2 + 1)((x-1)^2 + 1)
\equiv 0 \pmod p$, that is, $-4$ is a biquadratic residue.

\paragraph{Ex. 5.26}
This exercise and Ex. 5.27 and 5.28 give Dirichlet's beautiful proof
that $2$ is a biquadratic residue modulo $p$ iff $p$ can be written in
the form $A^2 + 64B^2$ , where $A, B \in \Z$. Suppose that $p \equiv 1
\pmod 4$. Then $p = a^2 + b^2$ by Ex. 5.24. Take $a$ to be odd.  Prove
the following statements:

\begin{enumerate}[a)]
\item $(a/p) = 1$.
\item $((a + b)/p) = (-1)^{((a+b)^2 -1)/8}$.
\item $(a + b)^2 \equiv 2ab \pmod p$
\item $(a + b)^{(p- 1)/2} \equiv (2ab)^{(p- 1)/4} \pmod p$.
\end{enumerate}

We'll begin with proving $(a/p) = 1$. Since $a$ is odd, we can use
quadratic reciprocity theorem with Jacobi symbols:

\begin{equation}
  \left(\frac{a}{p}\right)\left(\frac{p}{a}\right) = (-1)^{\frac{p-1}{2}\frac{a-1}{2}} = 1
\end{equation}

Thus $(a/p) = (p/a)$. On the other hand, as $p - b^2 = a^2$, $p \equiv
b^2 \pmod a$, and so $(p/a) = (b^2/a) = 1$, which proves a).

We now focus on b). We proceed similarly as in a): using quadratic
reciprocity, since $a+b$ is odd, we have $((a+b)/p) = (p/(a+b))$. On
the other hand, since $2p = (a+b)^2 + (a-b)^2$, we have $2p \equiv
(a-b)^2 \pmod {a+b}$, and so $(2p/(a+b)) = ((a-b)^2/(a+b)) = 1$. But
since $(2p/(a+b)) = (2/(a+b))(p/(a+b))$, $(p/(a+b)) = (2/(a+b))$, and
$(2/(a+b)) = (-1)^{\frac{(a+b)^2 - 1}{8}}$.

We proceed to c), which is pretty trivial: $(a+b)^2 = a^2 + b^2 + 2ab
= p + 2ab$, from which the congruence in c) follows immediately. The
congruence from d) also follows immediately from c).

\paragraph{Ex. 5.27}
Suppose that $f$ is such that $b \equiv af \pmod p$. Show that $f^2
\equiv -1 \pmod p$, and that $2^{(p-1)/4} \equiv f^{ab/2} \pmod p$.

If $b \equiv af \pmod p$. then $0 = (b-af)(b+af) = b^2 - f^2 a^2 \pmod
p$, so $b^2 \equiv f^2 a^2 \pmod p$. But from $p = a^2 + b^2$ we know
that $b^2 \equiv -a^2 \pmod p$, and so $-a^2 \equiv f^2 a^2 \pmod p$,
and since $a \not \equiv 0 \pmod p$, $-1 \equiv f^2 \pmod p$.

Combining b) with d) from the previous exercise, we get that:
\begin{equation}
(-1)^{(p-1+2ab)/8} = (-1)^{((a+b)^2 - 1)/8} = ((a+b)/p) \equiv
  (a+b)^{(p-1)/2} \equiv (2ab)^{(p-1)/4} \pmod p.
\end{equation}

For the left hand side, we have:
\begin{equation}
  (-1)^{(p-1 +2ab)/8} \equiv (f^2)^{(p-1+2ab)/8} = f^{(p-1 + 2ab)/4}
  \pmod p
\end{equation}

On the other hand, as $b \equiv af \pmod p$,
\begin{equation}
  (2ab)^{(p-1)/4} \equiv (2a^2 f)^{(p-1)/4} = (2f)^{(p-1)/4}
  a^{(p-1)/2} \equiv (2f)^{(p-1)/4} \pmod p,
\end{equation}
since $a$ is a quadratic residue mod p.

We thus get:
\begin{equation}
  f^{(p-1 + 2ab)/4} \equiv (2f)^{(p-1)/4} \pmod p
\end{equation}
and so dividing both sides by $f^{(p-1)/4}$, we get:
\begin{equation}
  f^{ab/2} \equiv 2^{(p-1)/4} \pmod p,
\end{equation}
which is a desired conclusion.

\paragraph{Ex. 5.28}
Show that $x^4 \equiv 2 \pmod p$ has a solution for $p \equiv 1 \pmod
4$ iff $p$ is of the form $A^2 + 64B^2$.

We note that $2$ is a biquadratic residue precisely when $2^{(p-1)/4}
\equiv 1 \pmod p$. By previous exercise, $2^{(p-1)/4} \equiv f^{ab/2}
\pmod p$, $a, b$ and $f$ are such that $p = a^2 + b^2$, $a$ is odd,
and $b \equiv af \pmod p$. Now since $f^2 \equiv -1 \pmod p$ by
previous exercise, $f^{ab/4} \equiv 1 \pmod p$ precisely when $ab/2$
is divisible by 4, but since $a$ is odd, it happens precisely when $b$
is divisible by $8$. Thus we get a conclusion: $2$ is a biquadratic
residue precisely when $b = 8B$, that is, when $p = A^2 + 64B^2$ for
$A = a$ and $B = b/8$.

\paragraph{Ex. 5.29}
Let $(RR)$ be the number of pairs $(n, n + 1)$ in the set $1,2,3,
\ldots, p - 1$ such that $n$ and $n + 1$ are both quadratic residues
modulo $p$. Let $(NR)$ be the number of pairs $(n, n + 1)$ in the set
$1,2,3, \ldots, p - 1$ such that $n$ is a quadratic nonresidue and $n
+ 1$ is a quadratic residue. Similarly, define $(RN)$ and
$(NN)$. Determine the sums $(RR) + (RN)$, $(NR) + (NN)$, $(RR) +
(NR)$, and $(RN) + (NN)$.

This is actually pretty obvious: consider $(RR) + (RN)$. For every
quadratic residue $n < p-1$, either $n+1$ is a quadratic residue, in
which case it contributes 1 to $(RR)$, or it isn't, in which case it
contributes 1 to $(RN)$. Thus $(RR)+(RN)$ is the number of quadratic
residues smaller than $p-1$, which is $(p-1)/2 = (p-2-(-1))/2$ when
$-1$ is a quadratic nonresidue, and $(p-1)/2-1 = (p-2-1)/2$ when -1 is
a quadratic residue. Since $-1$ is a quadratic residue precisely when
$\epsilon = (-1)^{(p-1)/2} = 1$, we can represent these to cases using
a single formula $(RR) + (RN) = (p-2-\epsilon)/2$.

The argunent in case $(NR) + (NN)$ is pretty much the same: every
nonresidue smaller than $p-1$ contributes 1 to the sum, so $(NR) +
(NN)$ is the number of nonresidues smaller than $p-1$, and so the
result is $(NR) + (NN) = (p-2+\epsilon)/2$.

For $(RR) + (NR)$ there's a little twist: if $n+1 > 1$ is a quadratic
residue, then if $n$ is a residue too, then it contributes one to
$(RR)$, and if it's not, it contributes one to $(NR)$. Thus, every
residue $n \geq 2$ contributes 1 to $(RR) + (NR)$. Since $1$ is always
a quadratic residue, there are exactly $(p-1)/2 - 1$ residues larger
than 1, so $(RR) + (NR) = (p-3)/2$.

On the other hand, the sum $(RN) + (NN)$ is always exactly $(p-1)/2$:
when considering pairs $(n, n+1)$, all nonresidues can always occur as
$n+1$.

Similar argument tells us that $(RR) + (RN) + (NR) + (NN)$ is exactly
$p-2$: there are $p-2$ pairs of the form $(n, n+1)$ with $n, n+1$ in
$1, 2, \ldots, p-1$, and each $(n, n+1)$ contributes to exactly one of
$(RR), (RN), (NR), (NN)$. On the other hand, if we add all the sums we
calculated beforehand, we get:

\begin{eqnarray}
2(p-2) &=& 2((RR) + (RN) + (NR) + (NN)) \\ &=& ((RR) + (RN)) + ((NR) +
(NN)) + ((RR) + (NR)) + ((RN) + (NN)) \\ &=& \frac{p-2 - \epsilon +
  p-2+\epsilon+p-3 + p-1}{2} \\ &=& \frac{4p - 8}{2} = 2p-4
\end{eqnarray}

And so it seems that we got it basically correct.

\paragraph{Ex. 5.30}
Show that $(RR) + (NN) - (RN) - (NR) =
\sum_{n=1}^{p-1}(n(n+1)/p)$. Evaluate this sum and show that it is
equal to $-1$.

Note there's a misprint in the book that says that the sum is
$\sum_{n=1}^{p-1} (n(n+1))/p$, so that we're not summing Legendre
symbols. In that case, once we evaluate the sum, we get the following:

\begin{eqnarray}
  \sum_{n=1}^{p-1} \frac{n(n+1)}{p} &=& \sum_{n=1}^{p-1} \frac{n^2 +
    n}{p} \\ &=& \frac{1}{p}\left(\frac{p(p+1)(2p+1)}{6} +
  \frac{p(p+1)}{2}\right) \\ &=& \frac{(p+1)(2p+1) + 3(p+1)}{6} \\ &=&
  \frac{(p+1)(2p+5)}{6}
\end{eqnarray}
I don't even see how this is an integer.

So, we'll show that the given sum is actually equal to
$\sum_{n=1}^{p-1}(n(n+1)/p)$, but this is pretty immediate:
$(n(n+1)/p) = (n/p)((n+1)/p)$, and this product equals 1 if either
both or none of $n, n+1$ are quadratic residue, and it equals $-1$ if
exactly one of them is a quadratic residue. Thus, each summand in
$\sum_{n=1}^{p-2}(n(n+1)/p)$ represent a single pair $(n, n+1)$, and
contributes exactly 1 to exactly one of $(RR), (NN), -(NR), -(RN)$. We
can easily extend the summation to $p-1$, as $((p-1)p/p) = 0$.

What remains now is to show that $\sum_{n=1}^{p-1}(n(n+1)/p) =
-1$. Don't know how to proceed. The hint to use Ex. 5.8 doesn't seem
too useful: we have here $\sum_{n=1}^{p-1}((n^2 + n)/p)$, and Ex. 5.8
helps us to deal with sums of the form $\sum_{n=1}^{p-1}((n^2 +
a)/p)$, where $a$ is kept constant.

\paragraph{Ex. 5.31}
Use the results of Ex. 5.29 and 5.30 to show that $(RR) =
\frac{1}{4}(p-4-\epsilon)$, where $\epsilon = (-1)^{(p-1)/2}$

We simply have a system of equations:

\begin{eqnarray}
  (RR) + (RN) &=& (p-2-\epsilon)/2 \\
  (NR) + (NN) &=& (p-2+\epsilon)/2 \\
  (RR) + (NR) &=& (p-3)/2 \\
  (RN) + (NN) &=& (p-1)/2 \\
  (RR) + (NN) - (RN) - (NR) &=& -1  \\
\end{eqnarray}

We should get the result by simply solving it, but I'm too lazy to
actually do it. Maybe another time. Pull requests welcome.

\paragraph{Ex. 5.32}
If $p$ is an odd prime, show that $(2/p) = \prod_{j=1}^{(p-1)/2} 2
\cos(2 \pi j/p)$. Use this to give another proof to Proposition 5.1.3.

Define:
\begin{equation}
  f(z) = 2 i \sin(2 \pi z)
\end{equation}
\begin{equation}
  g(z) = 2 \cos(2 \pi z)
\end{equation}

We need to show that:
\begin{equation}
(2/p) = \prod_{j=1}^{(p-1)/2} g(j/p)
\end{equation}

We note that:
\begin{eqnarray}
  \left(\prod_{j=1}^{(p-1)/2} g(j/p)\right)\left(\prod_{j=1}^{(p-1)/2} f(j/p)\right)
  &=& \prod_{j=1}^{(p-1)/2} f(j/p)g(j/p) \\
  &=& \prod_{j=1}^{(p-1)/2} 2i \cdot 2\sin(2 \pi j/p) \cos(2 \pi j/p) \\
  &=& \prod_{j=1}^{(p-1)/2} 2i \sin(2 \pi 2j/p) \\
  &=& \prod_{j=1}^{(p-1)/2} f(2j/p)
\end{eqnarray}
since $2 \sin(x) \cos(x) = \sin(2x)$.

On the other hand, using Proposition 5.3.2 from the book,
\begin{equation}
  \prod_{j=1}^{(p-1)/2} f(2j/p) = (2/p) \prod_{j=1}^{(p-1)/2} f(j/p)
\end{equation}
Thus:
\begin{equation}
  \prod_{j=1}^{(p-1)/2} f(j/p)g(j/p) = \prod_{j=1}^{(p-1)/2} f(2j/p) =
  (2/p) \prod_{j=1}^{(p-1)/2} f(j/p)
\end{equation}
From which the desired equality follows.

Now onto showing the quadratic character of 2. I am quite ashamed to
admit that I spent way too much time trying to figure this one
out. Initially I tried to calculate the product. It seems like a
classic trigonometric identity could be helpful:
\begin{equation}
\prod_{i = 1}^n \cos(\alpha_i) = \frac{1}{2^n}\left(\sum_{e \in S}
\cos(e_1 \alpha_1 + \ldots + e_n \alpha_n\right)
\end{equation}
where $S = \{-1, 1\}^n$. It is obtained by repeated application of:
\begin{equation}
  2 \cos(\theta)\cos(\phi) = \cos(\theta + \phi) + \cos(\theta - \phi)
\end{equation}
However, finally I realized that this is all wholly unnecessary: we
already know that the result will be $1$ or $-1$, and we only care
about the sign. The sign of a product is quite easy to determine: we
just need to calculate if the number of negative factors is odd or
even. In our case, we are multiplying real parts of first $(p-1)/2$
$p$-th roots of unity. These are all the roots that lie above the real
axis. The number of negative factors is thus exactly the number of
$p$-th roots of unity that lie in second quadrant. It's easy to see
that the parity here depends on what $p$ is modulo $8$.

If $p = 8k+1$, then there are exactly $4k$ roots that lie above real
axis, and exactly half of them, $2k$, lie in second quadrant. Since
$2k$ is even, we have even number of negative factors, and so $(2/p) =
1$.

When $p = 8k + 3$, we have $4k+1$ roots above real axis. The number
$4k+1$ is odd, and so we need to figure out if the middle root of
these $4k+1$ roots lies in first or second quadrant. Middle root is
$2k+1$-th one, and $(2k+1)/(8k+3) > (2k+1)/(8k+4) = 1/4$, thus middle
roots lies in second quadrant, and so we have $2k$ roots in first
quadrant and $2k+1$ roots in the second one. Therefore we have odd
number of negative factors, and so in this case $(2/p) = -1$.

Analysis of the remaining two cases is similar.

\paragraph{Exercise 5.33}
Use Proposition 5.3.2 to derive the quadratic character of $-1$.

This is a quite weird and roundabout way to answer this easy question,
but let's play along. The Proposition 5.3.2 in this case, setting $a =
-1$, tells us that:

\begin{equation}
  \prod_{l = 1}^{(p-1)/2} f\left(\frac{-l}{p}\right) =
  \left(\frac{-1}{p}\right) \prod_{l = 1}^{(p-1)/2}
  f\left(\frac{l}{p}\right)
\end{equation}

On the other hand, we note that $f\left(\frac{-l}{p}\right) =
-f\left(\frac{l}{p}\right)$, thus we have:

\begin{equation}
  \prod_{l = 1}^{(p-1)/2} f\left(\frac{-l}{p}\right) = (-1)^{(p-1)/2}
  \prod_{l = 1}^{(p-1)/2} f\left(\frac{l}{p}\right) =
  \left(\frac{-1}{p}\right) \prod_{l = 1}^{(p-1)/2}
  f\left(\frac{l}{p}\right)
\end{equation}
Dividing by $\prod_{l = 1}^{(p-1)/2} f\left(\frac{l}{p}\right)$ yields
the result.

\paragraph{Exercise 5.34}
If $p$ is an odd prime distinct from 3, show that
\begin{equation}
  \left(\frac{3}{p}\right) = \prod_{j=1}^{(p-1)/2} \left(3-4 \sin^2\left(\frac{2 \pi j}{p}\right)\right)
\end{equation}

We note that:
\begin{eqnarray}
  \sin(3x) &=& \cos(x)\sin(2x) + \cos(2x)\sin(x) \\
  &=& \cos(x) (2 \sin(x) \cos(x)) + (\cos^2(x) - \sin^2(x)) \sin(x) \\
  &=& 2 \sin(x)(1 - \sin^2(x)) + (1 - 2 \sin^2(x)) \sin(x) \\
  &=& 2 \sin(x) - 2 \sin^3(x) + \sin(x) - 2\sin^3(x) \\
  &=& \sin(x) (3 - 4 \sin^2(x))
\end{eqnarray}

Therefore,
\begin{equation}
  \prod_{j=1}^{(p-1)/2} \left(3-4 \sin^2\left(\frac{2 \pi
    j}{p}\right)\right) = \prod_{j=1}^{(p-1)/2} \frac{\sin
    \left(\frac{3\cdot2 \pi j}{p}\right)} {\sin\left(\frac{2 \pi
      j}{p}\right)} = \prod_{j=1}^{(p-1)/2} \frac{f
    \left(\frac{3j}{p}\right)} {f\left(\frac{j}{p}\right)}
\end{equation}
where
\begin{equation}
  f(z) = 2 i \sin(2 \pi z)
\end{equation}
The desired equality now follows from applying Proposition 5.3.2.

\paragraph{Exercise 5.35}
Use the preceding exercise to show that $3$ is a square modulo $p$ iff
$p$ is congruent to $-1$ or $-1$ modulo $12$.

The analysis here is similar to the one in Exercise 5.32. Here,
instead of determining the sign of $\cos(2 \pi j/p)$, we need to
determine the sign of $\left(3-4 \sin^2\left(\frac{2 \pi
  j}{p}\right)\right)$. Clearly, it's negative precisely when $
\sin^2\left(\frac{2 \pi j}{p}\right) > 3/4$. Conveniently, $3/4$ just
happens to be equal to $\sin^2(2 \pi/6)$, which again allows us to
divide the circle into now 6 parts, as opposed to 4 parts in Exercise
5.32, and then do the similar analysis based on what $p$ modulo $12$
is. 

\paragraph{Exercise 5.36}
Show that part (c) of Proposition 5.2.2 is true if $a$ is negative and
$b$ is positive (both still odd).

This question is ill-posed, as Jacobi symbol $(b/a)$ is undefined when
$a$ is negative.

\paragraph{Exercise 5.37}
Show that if $a$ is negative, then $p \equiv q (4a)$, $p \not | a$ implies $(a/p) = (a/q)$.

Write $b = -a$, then $b$ is positive, and $p \equiv q (4b)$. Note that
$(a/p) = (-b/p) = (-1)^{(p-1)/2} (b/p)$, and $(a/q) = (-b/q) =
(-1)^{(q-1)/2} (b/q)$. Moreover, $(-1)^{(p-1)/2} = (-1)^{(q-1)/2}$,
since $p \equiv q (4)$ (as $p \equiv q (4b)$). Thus, result follows
from Proposition 5.3.3.

\paragraph{Exercise 5.37}
Let $p$ be an odd prime. Derive the quadratic character of $2$ modulo
$p$ by verifying the following steps, involving the Jacobi symbol:

\begin{equation}
  \left(\frac{2}{p}\right) = \left(\frac{8-p}{p}\right) =
  \left(\frac{p}{p-8}\right) = \left(\frac{8}{p-8}\right) =
  \left(\frac{2}{p-8}\right)
\end{equation}
Generalize the argument to show that
\begin{equation}
  \left(\frac{a}{p}\right) = \left(\frac{a}{p-4a}\right)
\end{equation}

We'll prove the general statement. We'll assume that $p > 4a$, so that
$(a/(p-4a))$ is well defined. Since $4 = 2^2$ is always a quadratic
residue, $(a/p) = (4a/p)$. Now, since $4a \equiv p-4a (p)$, we have
\begin{equation}
  \left(\frac{a}{p}\right) = \left(\frac{4a}{p}\right) =
  \left(\frac{p-4a}{p}\right)
\end{equation}
Then we note that $p \equiv p-4a (4)$, and thus
\begin{equation}
  (-1)^{(p-1)/2} = (-1)^{(p-4a-1)/2}
\end{equation}
and so by quadratic reciprocity with Jacobi symbols,
\begin{equation}
  \left(\frac{p-4a}{p}\right) = \left(\frac{p}{p-4a}\right)
\end{equation}
Now, clearly $p \equiv 8 (p-8)$, hence
\begin{equation}
\left(\frac{p}{p-4a}\right) = \left(\frac{8}{p-4a}\right) =
\left(\frac{2 \cdot 4}{p-4a}\right) = \left(\frac{2}{p-4a}\right)
\end{equation}
and so we're done.

I spent quite a while trying to figure out why the problem asks us to
show that $(2/p) = ((8-p)/p)$. It seems like quite a strange thing to
show, with negative ``numerator'' of Jacobi symbol, especially as the
next step immediately jumps to $p-8$ in the ``denominator''. I have no
idea what was the idea here.

\section{Chapter 6}
\paragraph{Exercise 6.1}
Show that $\sqrt{2} + \sqrt{3}$ is an algebraic integer.

Clearly, both $\sqrt{2}$ and $\sqrt{3}$ are algebraic integer, and so
their sum also is, by Proposition 6.1.5.

\paragraph{Exercise 6.2}
Let $\alpha$ be an algebraic number. Show that there's an integer $n$
such that $n\alpha$ is an algebraic integer.

Since $\alpha$ is an algebraic number, we have $f(\alpha) = 0$ for
some polynomial $f(x) = a_n x^n + \ldots + a_0$. Let
\begin{equation}
g(y) = y^n + a_{n-1} y^{n-1} + a_{n-1} a_n y^{n-2} + a_{n-2} a_n^2 y^{n-3} + \ldots + a_1 a_n^{n-2} y + a_0 a_n^{n-1}
\end{equation}
Then $g(a_n \alpha) = 0$, since: 
\begin{eqnarray}
  g(a_n \alpha) &=& (a_n \alpha)^n + a_{n-1} (a_n \alpha)^{n-1} +
  a_{n-1} a_n (a_n \alpha)^{n-2} + \ldots + a_1 a_n^{n-2} a_n \alpha +
  a_0 a_n^{n-1} \\ &=& a_n^{n-1}(a_n \alpha^n + a_{n-1} \alpha^{n-1} +
  \ldots + a_1 \alpha + a_0) = 0
\end{eqnarray}
since $a_n \alpha^n + a_{n-1} \alpha^{n-1} + \ldots + a_1 \alpha + a_0
= f(\alpha) = 0$.

\paragraph{Exercise 6.3}
If $\alpha$ and $\beta$ are algebraic integers, prove that any
solution to $f(x) = x^2 + \alpha x + \beta = 0$ is an algebraic
integer. Generalize this result.

Let $\gamma_1, \gamma_2$ be the two solutions of $f(x) = 0$. Let $V =
\Z[\alpha, \beta, \gamma_1, \gamma_2, \gamma_1 \gamma_2,
  \alpha \gamma_1, \alpha \gamma_2, \beta \gamma_1, \beta \gamma_2,
  1]$. We note that $\gamma_i^2 = - \alpha \gamma_i - \beta \cdot 1
\in V$. Therefore, $\gamma_i V \subset V$, and so $\gamma_i$ are
algebraic integers.

Obviously, the generalization is that whenever $\gamma$ is a root of
$x^n + \alpha_{n-1} x^{n-1} + \ldots + \alpha_1 x + \alpha_0$, and
$\alpha_i$ are algebraic integers, so is $\gamma$. It can be proven by
the method above, but it's much easier to note that Proposition 6.1.4
implies that $\gamma$ is algebraic integer whenever we have $\Z
\subset \Z[\gamma] \subset W$ for some finitely generated $\Z$-module
$W$. We also note that if $\gamma$ is an algebraic integer, then
$\Z[\gamma]$ is finite $\Z$-module itself -- essentially because
$\gamma^n = -a_{n-1} \gamma^{n-1} - \ldots - a_0$. Thus, $\Z \subset
\Z[\alpha_0] \subset \ldots \subset \Z[\alpha_0, \ldots, \alpha_{n-1}]$
  is a chain of finite module extensions, and thus $B = \Z[\alpha_0,
    \ldots, \alpha_{n-1}]$ is also a finite $\Z$-module. Then,
  $B[\gamma]$ is finite over $B$, and so $B[\gamma]$ is finite over
  $\Z$.

\paragraph{Exercise 6.4}
A polynomial $f(x) \in \Z[x]$ is said to be primitive if the greatest
common divisor of its coefficients is 1. Prove that product of
primitive polynomials is also primitive.

Let $f, g \in \Z[x]$ be primitive. Let $h = fg$. Let $n$ be greatest
common divisor of the coefficients of $h$. Suppose $n \ne 1$. Let $p$
be a prime dividing $n$ Let $\overline{f}, \overline{g} \in \Z/(p)[x]$
be reductions of $f, g$ modulo $p$. Since $f$ and $g$ are primitive,
both $\overline{f}, \overline{g}$ are nonzero. Since $\Z/(p)[x]$ is a
domain, $\overline{f}\overline{g} \ne 0$. But
$\overline{f}\overline{g} = \overline{fg} = \overline{h} \ne 0$, and
so $p$ must not divide at least one of the coefficients of $h$, which
is a contradiction.

\paragraph{Exercise 6.5}
Let $\alpha$ be an algebraic integer and $f(x) \in \Q[x]$ be the monic
polynomial of least degree such that $f(\alpha) = 0$. Use Exercise 6.4
to show that $f(x) \in \Z[x]$.

Indeed, since $\alpha$ is an algebraic integer, there exists monic
$g(x) \in \Z[x]$ such that $g(\alpha) = 0$. Then, since $f$ is
minimal, it must divide $g$, and so $g = fh$ for some $h \in
\Q[x]$. Note that since $f$ and $g$ are both monic, $h$ must also be
monic.

Let $u$ be the least common multiple of the denominators of the
coefficients of $f$ in their lowest terms. Then $uf \in \Z[x]$ is
primitive: write $f(x) = x^n + (a_{n-1}/b_{n-1}) x^{n-1} + \ldots +
(a_0/b_0)$, then $u$ is the least common multiple of $b_0, \ldots,
b_{n-1}$. We have $uf(x) = u x^n + (u/b_{n-1}) a_{n-1} x^{n-1} +
\ldots + (u/b_0) a_0$. Let $p$ be any prime dividing $u$, and let
$p^k$ be the highest power of $p$ dividing $u$. Then $p^k$ must divide
one of the $b_i$-s -- indeed, otherwise $u/p$ would be a smaller
common multiple of $b_i$-s. Therefore, let $b_j$ be such that $p^k |
b_j$, then $p$ does not divide $u/b_j$. Obviously, $p$ does not divide
$a_j$ either, as it divides $b_j$, and $a_j/b_j$ is in lowest
terms. Thus $p$ does not divide $(u/b_j)a_j$, which is a coefficient
at $x^j$ in $uf(x)$.

We similarly define $v$ to be the least common multiple of the
denominators of the coefficients of $h$ in their lowest terms, and
similarly argue that $vg$ is primitive. We then have $uvg = (uf)(vg)$,
and since both $uf, vg$ are primitive, so is $uvg$, but that can only
happen when $u = v = 1$, in which case $f, h \in \Z[x]$.


\end{document}

