\documentclass[notitlepage]{article} 
\usepackage[MeX]{polski}
\usepackage[utf8]{inputenc} \usepackage{anysize}
\usepackage{graphicx}
\usepackage{enumerate} 
\usepackage{amsmath} 
\usepackage{amssymb} 
\usepackage{amsfonts}
\usepackage{amsthm}
\marginsize{3cm}{2cm}{1cm}{1cm}

\newtheorem{theorem}{Theorem}[section]
\newtheorem{lemma}[theorem]{Lemma}
\newtheorem{proposition}[theorem]{Proposition}
\newtheorem{corollary}[theorem]{Corollary}


\theoremstyle{definition}
\newtheorem*{definition}{Definition}
\newtheorem{example}{Example}
\newtheorem{remark}[theorem]{Remark}

\newcommand\Spec{\operatorname{Spec}}
\newcommand\Proj{\operatorname{Proj}}
\newcommand\Pic{\operatorname{Pic}}
\newcommand\rad{\operatorname{rad}}
\newcommand\rank{\operatorname{rank}}
\newcommand\im{\operatorname{im}}
\newcommand\id{\operatorname{id}}
\newcommand\sheafhom{\mathcal{H}om}
\newcommand\coker{\operatorname{coker}}
\newcommand\codim{\operatorname{codim}}
\newcommand\Hom{\operatorname{Hom}}
\newcommand\Ext{\operatorname{Ext}}
\newcommand\sspan{\operatorname{span}}
\newcommand\chark{\operatorname{char}}
\renewcommand\O{\mathcal{O}}
\renewcommand\P{\mathbb{P}}
\newcommand\A{\mathbb{A}}
\newcommand\C{\mathbb{C}}
\newcommand\I{\mathcal{I}}
\newcommand\Z{\mathbb{Z}}
\newcommand\N{\mathbb{N}}
\newcommand\E{\mathcal{E}}
\newcommand\F{\mathcal{F}}
\newcommand\GG{\mathbb{G}}
\newcommand\LL{\mathcal{L}}
\newcommand\p{\mathcal{P}}

\title{Solutions to Ireland, Rosen ``A Classical Introduction to Modern Number Theory''}
\author{Adam Michalik} 

\begin{document}
\maketitle
\section{Chapter 1}

\paragraph{Ex. 1.1}

Let $a$ and $b$ be nonzero integers. We can find nonzero integers $q$
and $r$ such that $a = qb + r$ where $0 \leq r < b$. Prove that $(a, b) = (b, r)$

As a reminder, $(a_1, \ldots, a_n)$ is defined to be the ideal
generated by $a_i$, but also sometimes by abuse of notation it denotes
the smallest positve member of the ideal (which generates it).

The relation $a = qb + r$ shows that $a \in (b, r)$, so $(a, b) \subset (b, r)$.
On the other hand, $r = qb - a$, so $r \in (a, b)$, whus $(r, b) \subset (a, b)$.

\paragraph{Ex. 1.2}

XXX Exercise statement missing.

The only thing here that needs proving is that the process finishes in
finitely many steps, but this is clear, as $r_i > r_{i+1} \geq 0$ by
construction, so $r_i$ is a decreasing sequence of real numbers, which
cannot be infinite.

\paragraph{Ex. 1.3}

Calculate $(187, 221)$, $(6188, 4709)$, $(314, 159)$.

Use the method from Ex. 1.2. Calculation omitted.

\paragraph{Ex. 1.4}

Let $d = (a, b)$. Show how one can use the Euclidean algorithm to find
numbers $m$ and $n$ such that $am + bn = d$.

The method from Ex. 1.2 produces a sequence $r_i$, such that $r_{k+1}
= d$, and $r_{k+2} = 0$, that is, $r_{k+1} | r_k$. We have
\begin{equation}
  \label{eq:eucl-1}
  r_{k-1} = q_{k+1} r_{k} + r_{k+1},
\end{equation}
so
\begin{equation}
  \label{eq:eucl-2}
  r_{k+1} = r_{k-1} - q_{k+1} r_{k}.
\end{equation}
Next, we have
\begin{equation}
  \label{eq:eucl-3}
  r_{k-2} = q_{k} r_{k-1} + r_{k},
\end{equation}
so
\begin{equation}
  \label{eq:eucl-4}
  r_{k} = r_{k-2} - q_{k} r_{k-1}.
\end{equation}
Substituting (\ref{eq:eucl-4}) back to (\ref{eq:eucl-2}) allows us to
express $d = r_{k+1}$ in terms of $r_{k-1}, r_{k-2}$. Continuing this
procedure will allow us to express $d$ in terms of $r_i, r_{i-1}$, and
finally in terms of $r_1, r_0$, which can be expressed in terms of $a$
and $b$.

\paragraph{Ex. 1.5}

Find $m$ and $n$ for the pairs $a$ and $b$ given in Ex 1.3

Skipped.

\paragraph{Ex. 1.6}

Let $a, b, c \in \Z$. Show that the equation
\begin{equation}
  \label{eq:linear-dioph}
  ax + by = c
\end{equation}
has solutions in integers iff $(a, b) |c$.

From $ax + by = c$ it instantly follows that any common divisor of $a$
and $b$ also divides $c$, so $(a, b) | c$. On the other hand, if $(a,
b)|c$, let $d = (a, b)$, and write $c = dd'$ for $d' \in
\Z$. Write $am + bn = d$ for $m, n \in
\Z$. Multiplying by $d'$ gives us $amd' + bnd' = dd' = c$. We
see that this gives a solution to (\ref{eq:linear-dioph}) -- set $x =
md', y = nd'$.

\paragraph{Ex. 1.7}

Let $d = (a, b)$ and $a = da'$ and $b = db'$. Show that $(a', b') = 1$.

Let $d' | a'$. Since $a = da'$, this means that $d'd | a$. Similarly,
if $d' | b'$, we show that that $d' d | b$. This gives us $d' = 1$ --
otherwise, $d' d$ would have been a greater common divisor of $a, b$.

\paragraph{Ex. 1.8}

Let $x_0$ and $y_0$ be a solution to $ax + by = c$. Show that all
solutions have the form $x = x_0 + t(b/d)$, $y = y_0 - t(a/d)$, where $d =
(a, b)$ and $t \in \Z$.

We have $a(x-x_0) + b(y-y_0) = 0$. Clearly, $a/d | a(x-x_0)$, so also
$a/d | b(y-y_0)$. Since $(b/d, a/d) = 1$ by previous exercise, we also
$(b, a/d) = 1$, therefore $a/d$ must divide $y-y_0$. Similarly, $b/d$
must divide $x-x_0$. Let $x-x_0 = t(b/d), y-y_0 = t'(a/d)$. We have
$at(b/d) + bt'(a/d) = 0$, so $t' = -t$.

\paragraph{Ex. 1.9}
Suppose that $u, v \in \Z$ and that $(u, v) = 1$. If $u | n$
and $v | n$, show that $uv | n$. Show that this is false if $(u, v) \ne 1$.

If $(u, v) = d \ne 1$, let $n = u (v/d) = (u/d) v$. Clearly $u | n$
and $v | n$. On the other hand, $uv > n$, so it cannot divide $n$.

Now let $(u, v) = 1$, and $u|n$, $v|n$. Write $n = au = bv$. Since
$v|n$, $v | au$. Since $(u, v) = 1$, by proposition 1.1.1, $v | a$,
that is, $a = a'v$. Thus, $n = a'vu$, so $vu | n$.

\paragraph{Ex. 1.10}
Suppose that $(u, v) = 1$. Show that $(u+v, u-v)$ is either 1 or 2.

Let $d = (u+v, u-v)$. Since $d | u+v$ and $d | u-v$, by taking sum and
difference, $d | 2u$ and $d | 2v$. Take any prime $p | d$, it also
divides both $2u$ and $2v$. If $p > 2$, by proposition 1.1.1, $p|u$
and $p|v$, which contradicts $(u, v) = 1$.

\paragraph{Ex. 1.11}
Show that $(a, a+k)|k$.

Let $d |a$, $d | a+k$. It also must divide their difference, that is, $a+k - a = k$.

\paragraph{Ex. 1.12}

Suppose that we take several copies of a regular polygon and try to fit them evenly
about a common vertex. Prove that the only possibilities are six equilateral triangles,
four squares, and three hexagons.

Consider an arrangment of $k$ $n$-gons evenly about a common
vertex. The sum internal angle in a regular n-gon is $\pi - 2\pi/n =
(n-2)\pi/n$. Thefore, we must have $2\pi = k(n-2)\pi/n$, so $2n =
k(n-2)$, thus $2n + 2k = kn$. By symmetry, we can assume that $n \leq
k$. Then, $kn = 2k + 2n \leq 2k + 2k = 4k$, so $n \leq 4$, and thus $n
= 3$ and $k = 6$, or $n = 4$ and $k = 4$. From the symmetrical case,
we get $k = 3$, $n = 6$.

\paragraph{Ex. 1.13}
Skipped.

\paragraph{Ex. 1.14}
Skipped.

\paragraph{Ex. 1.15}
Prove that $a \in \Z$ is the square of another integer iff
$ord_p(a)$ is even for all primes $p$.  Give a generalization.

A generalization is of course ``$a$ is $n$-th power if $ord_p(a)$ is
divisible by $n$ for all primes $p$. There's a slight error in the
question -- one also needs to look at the sign of $a$ -- $-4$ is not a
square of another integer, even though $ord_p(a)$ is even for all
primes $p$. We'll therefore assume that $a \in \mathbb{N}$.

The exercise is obvious once we look at the unique factorization -- we have:
\begin{equation}
  a = \prod_{p|a,\, p\,\textrm{prime}} p^{ord_p(a)}
\end{equation}
Since $ord_p(a)$ are even, $ord_p(a)/2$ are integers, so:
\begin{equation}
  a = \prod_{p|a,\, p\,\textrm{prime}} (p^{ord_p(a)/2})^2 = \left(\prod_{p|a,\, p\,\textrm{prime}} p^{ord_p(a)/2}\right)^2
\end{equation}
For the other direction, if $a = b^2$, then clearly $ord_p(a) =
2ord_p(b)$. The proof generalizes for $n > 2$.

\paragraph{Ex. 1.16}
If $(u, v) = 1$ and $uv = a^2$ , show that both $u$ and $v$ are squares.

By previous exercise, and by symmetry, it's enough to prove that
$ord_p(u)$ is even. Now, $ord_p(a^2) = ord_p(u) + ord_p(v)$, so $2
ord_p(a) = ord_p(u) + ord_p(v)$. One of $ord_p(u), ord_p(v)$ must be
$0$, otherwise $p$ divides both $u$ and $v$, which contradicts $(u, v)
= 1$. Thus, either $ord_p(u) = 0$, which is even, or $ord_p(u) = 2
ord_p(a)$, which is even too.

\paragraph{Ex. 1.17}
Prove that the square root of 2 is irrational, i.e., that there is no rational number
$r = a/b$ such that $r^2 = 2$.

Follows from Ex 1.18

\paragraph{Ex. 1.18}
Prove that $\sqrt[n]{m}$ is irrational if $m$ is not the $n$-th power of an integer.

Let $r = a/b$ be such that $r^n = m$. Assume $r$ is in lowest terms,
that is, $(a, b) = 1$. It necessarily follows that $(a^n, b^n) = 1$,
so $m = r^n = a^n/b^n$ is also in lowest terms. Thus, if $m$ is an
integer, $b^n = 1$, so $b = 1$, and $r$ is also an integer.

\paragraph{Ex. 1.19}
Define the least common multiple of two integers $a$ and $b$ to be an integer $m$ such that
$a|m$, $b|m$, and $m$ divides every common multiple of $a$ and $b$. Show that such an $m$
exists. It is determined up to sign. We shall denote it by $[a, b]$.

Consider a set $I = \{n \in \Z: a|n \land b|n\}$. Clearly it
is an ideal of $\Z$, and so $I = (m)$ for some $m$.

\paragraph{Ex. 1.20} Skipped.
\paragraph{Ex. 1.21} Skipped.
\paragraph{Ex. 1.22} Skipped.

\paragraph{Ex. 1.23} 
Suppose that $a^2 + b^2 = c^2$ with $a, b, c \in \Z$ For example, $3^2 + 4^2 = 5^2$ and $5^2 +
12^2 = 13^2$ . Assume that $(a, b) = (b, c) = (c, a) = 1$. Prove that there exist integers $u$
and $v$ such that $c - b = 2u^2$ and $c + b$ = $2v^2$ and $(u, v) = 1$ (there is no loss in
generality in assuming that $b$ and $c$ are odd and that $a$ is even). Consequently $a = 2uv$,
$b = v^2 - u^2$ , and $c = v^2 + u^2$ . Conversely show that if $u$ and $v$ are given, then the
three numbers $a$, $b$, and $c$ given by these formulas satisfy $a^2 + b^2 = c^2$.

If $a^2 + b^2 = c^2$, we have $a^2 = c^2 - b^2 = (c-b)(c+b)$. By Ex
1.10, $(c-b, c+b)$ is either $1$ or $2$. Since both $b, c$ are odd,
both $c-b, c+b$ are even, and so $(c-b, c+b) = 2$. Thus, $c-b = 2x$,
$c+b = 2y$, and $(x, y) = 1$. Consider any prime $p \ne 2$. As $a^2 =
c^2 - b^2 = (c-b)(c+b)$, and $ord_p(c-b) = ord_p(x), ord_p(c+b) =
ord_p(y)$ for $p \ne 2$, it follows that $2 ord_p(a) = ord_p(c-b) +
ord_p(c+b) = ord_p(x) + ord_p(y)$, and one of the $ord_p(x), ord_p(y)$
must be $0$, as $(x, y) = 1$. Thus $ord_p(x)$ and $ord_p(y)$ are even
for all $p$, and so $x = u^2$, $y = v^2$.

The other direction is simple calculation.

\paragraph{Ex. 1.24} Skipped.

\paragraph{Ex. 1.25}
If $a^n - 1$ is a prime, show that $a = 2$ and that $n$ is a
prime. Primes of the form $2^p - 1$ are called Mersenne primes. For
example, $2^3 - 1 = 7$ and $2^5 - 1 = 31$. It is not known if there are
infinitely many Mersenne primes.

Let $a^n - 1$ be prime. We have:

\begin{equation}
  a^n - 1 = (a-1)(a^{n-1} + a^{n-2} + \ldots + a + 1)
\end{equation}

It follows that $a-1|a^n - 1$, but as $a^n - 1$ is prime, $a-1 = 1$,
and so $a = 2$.

Similarly, let $n = uv$, and let $u \leq v$. We have:
\begin{equation}
  2^n - 1 = 2^{uv} - 1 = (2^{u})^v - 1 = (2^u-1)((2^u)^{v-1} + (2^u)^{v-2} + \ldots + 2^u + 1)
\end{equation}
We thus have that $2^u - 1 | 2^n - 1$, but since $2^n - 1$ is prime,
$2^u -1 = 1$, so $u = 1$, and therfore $n$ is prime.

\paragraph{Ex. 1.26}
If $a^n + 1$ is a prime, show that $a$ is even and that $n$ is a power
of 2. Primes of the form $2^{2^t} + 1$ are called Fermat primes. For
example, $2^{2^1} + 1 = 5$ and $2^{2^2} + 1 = 17$.  It is not known if
there are infinitely many Fermat primes.

Let $a^n + 1$ be prime. If $a$ is odd, $a^n + 1$ is even, so it is
prime only if $a = n = 1$ (this is a minor mistake in the statement of
the problem). Assume now that $a$ is even. Suppose that $p | n$, $p$
is odd. Then, letting $n = pv$ for some $v$, we have:

\begin{equation}
  a^n + 1 = a^{pv} + 1 = (a^v)^p + 1 = (a^v+1)((a^u)^{p-1} + (a^v)^{p-2} + \ldots + a^v + 1)
\end{equation}

So, $a^v + 1|a^n + 1$, and therefore $a^n + 1$ is not prime.

\paragraph{Ex. 1.27}
For all odd $n$ show that $8| n^2 - 1$. If $3$ doesn't divide $n$,
show that $6| n^2 - 1$.

We have $n^2 - 1 = (n+1)(n-1)$. Since $n$ is odd, both $n+1, n-1$ are
even, and moreso, one of these must be divisible by 4, as one of the
two consecutive odd numbers is divisible by 4. Thus, their product is
divisible by $8$. Similarly, if $3$ does not divide $n$, it must
divide one of $n-1, n+1$, otherwise it wouldn't divide three
consecutive integers, which is impossible. As $n$ is odd, $n+1$ is
even, so $(n+1)(n-1)$ is divisible by both $2$ and $3$, so it is
divisible by $6$.

\paragraph{Ex. 1.28}
For all $n$ show that $30 | n^5 - n$ and that $42| n^7 - n$.

The reasoning here is very similar to the previous exercise -- for the
first, we use $30 = 2\cdot3\cdot5$ with $n^5 - n = n(n^4 - 1) =
n(n^2-1)(n^2+1) = n(n-1)(n+1)(n^2 + 1)$. It necessarily follows that
one of $n-1, n, n+1$ is divisible by 2, one by 3, and if none of them
is divisible by 5, it follows that $n$ gives the remainder of 2 or 3
when divided by 5. But then, $n^2$ gives the rest of $2^2 = 4$ in the
first case, and $3^2 \mod 5 = 4$ in the second case, so in both cases
$n^2 + 1$ is divisible by 5. We omit the similar argument for $42| n^7
- n = n(n^6-1) = n(n^2 - 1)(n^2 + n + 1)$.

\paragraph{Ex. 1.29}
Suppose that $a, b, c, d \in \Z$ and that $(a, b) = (c, d) =
1$. If $(a/b) + (c/d) =$ an integer, show that $b = d$ or $b = -d$.

We have:

\begin{equation}
  \frac{a}{b} + \frac{c}{d} = \frac{ad + bc}{bd}
\end{equation}

Assume that it is an integer, that is, $bd | ad + bc$. Since $d|bd$
and $d|ad$, we also have $d|bc$, but $(c, d) = 1$, so $d|b$. Similarly
we argue that $b|d$. The thesis follows.

\paragraph{Ex. 1.30}
Prove that

\begin{equation}
  H_n = 1+ \frac{1}{2} + \frac{1}{3} + \ldots + \frac{1}{n}
\end{equation}

is not an integer.

Let $2^s$ be the largest power of $2$ occuring as a denominator in
$H_n$, say $2^s = k \leq n$. Write $H_n = \frac{1}{2^s} + (1 + 1/2 +
\ldots + 1/(k-1) + 1/(k+1) + \ldots + 1/n$. The sum in parentheses can
be written as $1/2^{s-1}$ times sum of fractions with odd
denominators, so the denominator of the sum in parentheses will not be
divisible by $2^s$, but it must equal $2^s$ by Ex 1.29.

\paragraph{Ex. 1.31}
Show that $2$ is divisible by $(1+i)^2$ in $\Z[i]$.

We have $(1+i)^2 = 1 + 2i -1 = 2i$, so $2 = -i(1+i)^2$.

\paragraph{Ex. 1.32}
For $\alpha = a + bi \in \Z[i]$ we defined $\lambda(\alpha) =
a^2 + b^2$. From the properties of $\lambda$ deduce the identity $(a^2
+ b^2)(c^2 + d^2) = (ac - bd)^2 + (ad + bc)^2$.

It is not clear what exactly properties of $\lambda$ they mean, but
most likely the fact that $\lambda((ab) = \lambda(a)\lambda(b)$, which
is never stated in the text, but amounts to proving this very
identity.

\paragraph{Ex. 1.33}
Show that $\alpha \in \Z[i]$ is a unit iff  $\lambda(\alpha) = 1$. Deduce that 1, -1, i, and - i are the only units in $\Z[i]$.

If $\lambda{\alpha} = 1$, then we must have $a^2 + b^2 = 1$, and so
either $a^2 = 1, b^2 = 0$, in which case either $a = 1$ or $a = -1$,
or $a^2 = 0, b^2 = 1$, in which case $b = 1$ or $b = -1$. These 4
options give us $1, -1, i, -i$, all of which are units of $\Z[i]$.

In the other direction, let $\alpha$ be a unit, that is, there exists
$\beta$ such that $\alpha \beta = 1$. We have $1 = \lambda(1) =
\lambda(\alpha\beta) = \lambda(\alpha)\lambda(\beta)$, and so
$\lambda(\alpha) = 1$.

\paragraph{Ex. 1.34}
Show that 3 is divisible by $(1 - \omega)^2$ in $\Z[\omega]$.

We have $(1 - \omega)^2 = 1 - 2\omega + \omega^2$. Now, $\omega =
(-1+\sqrt{-3})/2$ and $\bar{\omega} = \omega^2$, so $1 - 2\omega +
\omega^2 = 2 - \sqrt{-3}+(-1-\sqrt{-3})/2 = 3(1-\sqrt{3})/2 =
3\cdot(-\omega)$. Now, $\omega$ is a unit of $\Z[\omega]$, so
$3 = -(1 - \omega)^2 \cdot \omega^{-1}$

\paragraph{Ex. 1.34}
For $\alpha = a + b\omega \in \Z[\omega]$ we defined
$\lambda(\alpha) = a^2 - ab + b^2$. Show that $\alpha$ is a unit iff
$\lambda(\alpha) = 1$. Deduce that $1, -1, \omega, -\omega, \omega^2 ,
and -\omega^2$ are the only units in $\Z[\omega]$.

It is enough to show that $\lambda$ is multiplicative. We have:

\begin{equation}
  \alpha = a + b\omega = (2a-b)/2 + ib\sqrt{3}/2
\end{equation}

Thus $|\alpha|^2 = (4a^2 - 4ab +b^2)/4 + 3b^2/4 = a^2 - ab + b^2$, so
$\lambda$ coincides with square of complex absolute value, which is
multiplicative.

\paragraph{Ex. 1.39}
Show that in any integral domain a prime element is irreducible.

Let $p$ be a prime element of $A$, that is, $(p) \subset A$ is a prime
ideal. Suppose $p = ab$. Since $ab \in (p)$, and $(p)$ is prime,
either $a$ or $b$ is in $(p)$, say $a$. Then $a = px$ for some $x \in
A$. We then have $p = ab = pxb$, so $p(1-bx) = 0$. Since $A$ is
integral domain, $1-bx = 0$, so $1 = bx$, that is, $b$ is a unit, and
therefore $p$ is irreducible.

\section{Chapter 2}

\paragraph{Ex. 2.1}
Show that $k[x]$, with $k$ a finite field, has infinitely many
irreducible polynomials.

Let $f_1, \ldots, f_n$ be a finite set of polynomials in
$k[x]$. Consider $f = 1+ \prod f_i$. It is not divisible by any of
$f_i$, so none of its irreducible factors can be equal to any of the
$f_i$. Therefore $f_1, \ldots, f_n$ is not the list of all irreducible
polynomials in $k[x]$.

\paragraph{Ex. 2.2}

The ring is just a localization of $\Z$ at $\prod(p_i) = (\prod p_i)$. This
corresponds to affine subscheme of $\Spec \Z$ consisting of points
${(p_i)}$.

\paragraph{Ex. 2.3}
Use the formula for $\phi(n)$ to give a proof that there are infinitely many primes.

Let $p_1, \ldots, p_t$ be all primes. Consider $n = \prod p_i$. Then
$\phi(n) = n\prod(1-1/p_i) = n \prod(p_i - 1)/p_i = \prod(p_i -
1)$. Since 3 is prime, $\phi(n) > 1$, but this means that there exists
$1 \leq k < n$ that is relatively prime to $n$, but this is
impossible, as any of its prime factors must also be a factor of $n$.

\paragraph{Ex. 2.4}
If $a$ is a nonzero integer, then for $n > m$ show that $(a^{2^n} + 1, a^{2^m} + 1) = 1$ or $2$
depending on whether $a$ is odd or even.

First we'll prove that if a prime $p$ divides $a^{2^m} + 1$, it must
also divide $a^{2^n} -1$ for all $n > m$. This is simple induction:
for $n = m+1$, we have $p|a^{2^m} + 1|(a^{2^m}+1)(a^{2^m}-1) =
(a^{2^m})^2 - 1 = a^{2^{m+1}} - 1$. The induction step is similar.

Thus, if $p$ divides both $a^{2^n} + 1, a^{2^m} + 1$ for $n > m$, it
also divides $a^{2^n} - 1$ by reasoning above, so it must divide the
difference $a^{2^n} + 1 - (a^{2^n} - 1) = 2$. This means that $p$ must
be $2$. It is easy to see that this will be the case whenever $a$ is
odd -- in that case, both $a^{2^n} + 1, a^{2^m} + 1$ will be even.

\paragraph{Ex. 2.5}
Use the result of Ex. 2.4 to show that there are infinitely many primes.

Clearly, since $(2^{2^n} + 1, 2^{2^m} + 1) = 1$, all of the prime
factors of $2^{2^n} + 1$ are different from all of the prime factors
of $2^{2^m} + 1$ for $n \ne m$ Since there are infinitely many numbers
of the form $2^{2^n} + 1$, there must be infinitely many primes.

\paragraph{Ex. 2.6}
For a rational number $r$ let $[r]$ be the largest integer less than
or equal to $r$, e.g., $[1/2] = 0$, $[2] = 2$, and $[3+1/3] =
3$. Prove $ord_p n! = [n/p] + [n/p^2] + [n/p^3] + \ldots$.

Since $n!$ is a product of $1, 2, 3, \ldots, n$, every $p$-th integer
contribute a factor of $p$ to the whole product, which correspond to
the $[n/p]$ summand in the formula, every $p^2$-th factor contributes
another $p$ factor to the product, which corresponds to $[n/p^2]$
summand, etc.

\paragraph{Ex. 2.7}
Deduce from Ex. 2.6 that $ord_p n! \leq n/(p - 1)$ and that
$\sqrt[n]{n!} \leq \prod_{p|n!} p^{1/(p-1)}$.

We have:
\begin{equation}
ord_p n! = [n/p] + [n/p^2] + [n/p^3] + \ldots \leq n/p + n/p^2 +
\ldots = \frac{n}{p}\sum_{i = 0} 1/p^i = \frac{n}{p}\frac{1}{1-1/p} = \frac{n}{p-1}
\end{equation}

The inequality $\sqrt[n]{n!} \leq \prod_{p|n!} p^{1/(p-1)}$ is
equivalent to $n! \leq \prod_{p|n!} p^{n/(p-1)}$, which is easily
derived by using the previous inequality to an equality:

\begin{equation}
  m = \prod_{p|m} p^{ord_p m}
\end{equation}

\paragraph{Ex. 2.7}
Use Exercise 7 to show that there are infinitely many primes.

Let $p_1, \ldots, p_n$ be all primes. Then $\sqrt[n]{n!} \leq
\prod_{p|n!} p^{1/(p-1)} \leq \prod_i p_i^{1/(p_i - 1)}$, which is
independent of $n$. This implies that $\sqrt{n}{n!}$ is a bounded
sequence. However, $(n!)^2 \geq n^n$ -- this is seen by noting that
$(n!)^2 = \prod_{1 \leq i \leq n} i(n-i+1)$, and for $1 \leq i \leq
n$, $i(n-i+1) \geq n$ - indeed, a quadratic function $-i^2 + i(n+1)$
attains maximum for $i = (n+1)/2$, and is monotonically decreasing in
both directions, while still being no smaller than $n$ for both $i =
1$ and $i = n$.

Therefore, $\sqrt[n](n!) \geq \sqrt{n}$, but $\sqrt{n}$ is unbounded,
which is a contradiction with boundedness of $\sqrt[n](n!)$.

\paragraph{Ex. 2.8}
A function on the integers is said to be multiplicative if $f(ab) =
f(a)f(b)$. whenever $(a, b) = 1$. Show that a multiplicative function
is completely determined by its value on prime powers.

Trivial.

\paragraph{Ex. 2.9}
If $f(n)$ is a multiplicative function, show that the function $g(n) =
\sum_{d|n} f(d)$ is also multiplicative.

We'll prove a stronger theorem, that is, if $f$ and $g$ are
multiplicative functions, then their Dirichlet product is also a
multiplicative function. 

Indeed, for $(a, b) = 1$:
\begin{equation}
  (f \circ g(a)) \cdot (f \circ g(b)) = \left(\sum_{d_1 d_2 = a}
  f(d_1)g(d_2)\right)\left(\sum_{d_3 d_4 = b} f(d_3)g(d_4)\right) =
  \sum_{d_1 d_2 = a, d_3 d_4 = b} f(d_1 d_3) g(d_2 d_4)
\end{equation}

Now we just need to convince ourselves that this is equivalent to
$\sum_{u_1 u_2 = ab} f(u_1)g(u_2)$, but this is clear: since $(a, b) =
1$, if $u_1 u_2 = ab$, $u_1$ can be uniquely factored into $d_1 d_3$
such that $d_1 | a, d_3 | b$, and same for $u_2$.

\paragraph{Ex. 2.10}
Show that $\phi(n) = n \sum_{d|n}\mu(d)/d$ by first proving that
$\mu(d)/d$ is multiplicative and then using Ex. 2.9 and 2.10.

The function $\mu{d}/d$ is multiplicative, as it's a pointwise product
of two multiplicative functions, $\mu(d)$ and $1/d$. Therefore, by
Ex. 2.10, $\sum_{d|n}\mu{d}/d$ is also multiplicative, and so is
$n\sum_{d|n}\mu{d}/d$, as a pointwise product of multiplicative
functions (obviously $n$ is multiplicative).  Let $f(n) = n
\sum_{d|n}\mu{d}/d$. As $f$ is multiplicative, it is fully determined
by its values on prime powers. If we show that $f(p^n) = \phi(p^n)$,
it implies that $f = \phi$.

The only $1 \leq i \leq p^n$ that aren't relatively prime with $p$ are
multiples of $p$. These are $p, 2p, 3p, \ldots, p^{n-1} p^n$. There
are exactly $p^{n-1}$ elements on this list, so $\phi(p^n) = p^n -
p^{n-1}$.

On the other hand, $f(p^n) = p^n \sum_{d|p^n}\mu(d)/d$. The only
$d|p^n$ such that $\mu(d) \ne 0$ is $d = 1$ and $d = p$. Thus, $f(p^n)
= p^n \sum_{d|p^n}\mu{d}/d = p^n(1 - 1/p)$, and so $f(p^n) =
\phi(p^n)$.

\paragraph{Ex. 2.12}
Find formulas for $f(n) = \sum_{d|n} \mu(d)\phi(d)$, $g(n) = \sum_{d|n}
\mu(d)^2\phi(d)^2$, and $h(n) = \sum_{d|n} \mu(d)/\phi(d)$.

All of the functions are multiplicative, by Ex. 2.9. Let's determine their values on prime powers.

We have:
\begin{equation}
  f(p^n) = \sum_{p^k|p^n} \mu(p^k) \phi(p^k) = \phi(1) - \phi(p) = -1 - p + 1 = -p
\end{equation}
Thus $f(n) = (-1)^k \prod_{i=1}^k p_i$, where $p_i$ are distinct prime factors of $n$.

\begin{equation}
  g(p^n) = \sum_{p^k|p^n} \mu(p^k)^2 \phi(p^k)^2 = \phi(1)^2 + \phi(p)^2 = 1 + p^2 - 2p + 1 = p^2 - 2p + 2
\end{equation}
Formula for $g(n)$ easily follows, but doesn't seem to be interesting.

\begin{equation}
  f(p^n) = \sum_{p^k|p^n} \mu(p^k)/ \phi(p^k) = 1/\phi(1) - 1/\phi(p)
  = -1 - \frac{1}{p - 1} = \frac{-p}{p-1}
\end{equation}

\paragraph{Ex. 2.13}
Let $\sigma_k(n) = \sum_{d|n} d^k$ . Show that $\sigma_k(n)$ is
multiplicative and find a formula for it.

It is clearly multiplicative, since $f(n) = n^k$ is multiplicative. We have:

\begin{equation}
  \sigma_k(p^n) = \sum_{d|p^n} d^k = 1 + p^k + (p^2)^k + \ldots + (p^n)^k = 1 + p^k + (p^k)^2 + \ldots + (p^k)^n = \frac{1-(p^k)^{n+1}}{1-p^k}
\end{equation}

\paragraph{Ex. 2.14}
If $f(n)$ is multiplicative, show that $h(n) = \sum_{d|n} \mu(n/d)f(d)$ is also multiplicative.

It follows from our solution to Ex. 2.9, as $\mu$ is multiplicative.

\paragraph{Ex. 2.15}
Show that
\begin{enumerate}[a]
  \item $\sum_{d|n} \mu(n/d)\nu(d) = 1$ for all n.
  \item $\sum_{d|n} \mu(n/d) \sigma(n) = n$ for all n.
\end{enumerate}

Both of the left hand sides are multiplicative functions of $n$, so it's enough to determine their value on prime powers. We have:

\begin{equation}
  \sum_{d|p^n} \mu(p^n/d)\nu(d) = \mu(p^n/p^{n-1}) \nu(p^{n-1}) +
  \mu(p^n/p^n) \nu(p^n) = \mu(p)\nu(p^{n-1}) + \mu(1) \nu(p^n) = -n +
  n+1 = 1
\end{equation}
since for $d$ other than $p^{n-1}$ and $p^n$, $\mu(p^n/d)$ is 0.

For (b),

\begin{eqnarray}
  \sum_{d|p^n} \mu(p^n/d)\sigma(d) &=& \mu(p^n/p^{n-1}) \sigma(p^{n-1})
  + \mu(p^n/p^n) \sigma(p^n) \\
  &=& \mu(p)\sigma(p^{n-1}) + \mu(1) \sigma(p^n) \\
  &=& -\frac{p^n -1}{p-1} + \frac{p^{n+1} -1}{p-1} \\
  &=& \frac{1 - p^n -1 + p^{n+1}}{p-1} = \frac{p^n(p-1)}{p-1} \\
  &=& p^n
\end{eqnarray}

\paragraph{Ex. 2.16}
Show that $\nu(n)$ is odd iff $n$ is a square.

This follows immediately from the formula:

\begin{equation}
  \nu(\prod p_i^{a_i}) = \prod(a_i+1)
\end{equation}

\paragraph{Ex. 2.17}
Show that $\sigma(n)$ is odd iff $n$ is a square or twice a square.

We have:
\begin{equation}
  \sigma(n) = \sigma(\prod p_i^{a_i}) = \prod_i\sum_{j=0}^{a_i}p^j
\end{equation}

For $\sigma(n)$ to be odd, it is necessary and sufficient that each of
the factors $\sum_{j=0}^{a_i}p^j$ is odd. For odd $p$, if $a_i$ even,
then $\sum_{j=0}^{a_i}p^j$ has odd number of odd summands, and
therefore is odd. For $p = 2$, the sum is always odd. The thesis now
follows.

\paragraph{Ex. 2.18}
Prove that $\phi(n)\phi(m) = \phi((n, m))\phi([n, m])$.

We have:
\begin{equation}
  \label{eq:phi-gcd-lcm}
  \phi([n, m]) = \phi(nm/(n,m)) = \phi(n/(n, m))\phi(m) = \phi(n)\phi(m/(n,m))
\end{equation}

Now $(n, m)$ must be relatively prime with one of $n/(n, m)$,
$m/(n,m)$ -- otherwise, if there was a prime divisor $p$ of $(n, m)$
that was also a prime divisor of both $n/(n, m), m/(n,m)$, it would
mean that $p(n, m)$ divides both $n$ and $m$, and so $(n, m)$ would
not be a greatest common divisor of $n$ and $m$. Without loss of
generality we can assume that $(n, m)$ is relatively prime with
$n/(n,m)$. Multiplying (\ref{eq:phi-gcd-lcm}) by $\phi((n, m))$ we
get:

\begin{equation}
  \phi((n, m))\phi([n, m]) = \phi((n, m))\phi(n/(n, m))\phi(m) =
  \phi((n,m) \cdot n/(n, m))\phi(m) = \phi(n)\phi(m)
\end{equation}
which is what we wanted to show.

\paragraph{Ex. 2.19}
Prove that $\phi(nm)\phi((n, m)) = (n, m)\phi(n)\phi(m)$.

\paragraph{Ex. 2.20}
Prove that $\prod_{d|n} d = n^{\nu(n)/2}$.

%% If $\nu(n)$ is odd, then $n = m^2$ by Ex. 2.16. The thesis is then
%% equivalent to $\prod_{d|m^2} d = (m^2)^{\nu(n)/2} = m^{\nu(n)}$. We use
%%   the Moebius inversion formula:
%% \begin{equation}
%%   n = \prod_{d|n} \prod_{e|d} e^{\mu(e)}
%% \end{equation}

\paragraph{Ex. 2.21}

Define $\land(n) = \log p$ if $n$ is a power of $p$ and zero
otherwise. Prove that $\sum_{d|n} \mu(n/d)\log d = \land (n)$.

Let $f(n) = \sum_{d|n} \land(d)$. Clearly, this is the same as $f(n) =
\sum_{p^k|n} \log p$, and there are exactly $ord_p n$ summands equal
to $\log p$. Thus, $f(n) = \log \prod_{p^k|n} p = \log \prod_{p|n}
p^{ord_p n} = \log n$. The equality now follows from Moebius inversion
formula.

\paragraph{Ex. 2.22}
Show that the sum of all the integers $t$ such that $1 \leq t \leq n$ and $(t,
n) = 1$ is $\frac{1}{2} n \phi(n)$.

Let $f(n)$ be the sum in question. Consider fractions $1/n, 2/n,
\ldots, n/n$, and reduce them to lower terms. The possible
denominators are all $d|n$. Consider the sum of numerators of
fractions with denominators equal to $d$. These are all $1 \leq t \leq
d$ with $(t, d) = 1$, and so the sum will be equal to $f(d)$. They all
come from some fractions $k/n$ by dividing the numerator and
denominator by $n/d$, so the sum of all $k$-s in the fractions $k/n$
that reduce to a fraction with denominator $d$ will be exactly $n/d
f(d)$. Therefore, the following equality holds:

\begin{equation}
  \sum_{d|n} \frac{n}{d} f(d) = \frac{n(n+1)}{2}
\end{equation}
which is equivalent to:
\begin{equation}
  \sum_{d|n} \frac{f(d)}{d} = \frac{n+1}{2}
\end{equation}

We apply Moebius inversion formula:
\begin{equation}
  \frac{f(n)}{n} = \sum_{d|n}\mu\left(\frac{n}{d}\right) \frac{d+1}{2} =
  \frac{1}{2}\left(\sum_{d|n}\mu\left(\frac{n}{d}\right)d +
  \sum_{d|n}\mu\left(\frac{n}{d}\right)\right)
\end{equation}

The second sum is 0, so we are left with proving that
$\sum_{d|n}\mu\left(\frac{n}{d}\right)d = \phi(n)$, but this follows
instantly by applying Moebius inversion formula to the equality $n =
\sum_{d|n} \phi(d)$.

\paragraph{Ex. 2.23}
Let $f(x) \in \Z[x]$ and let $\psi(n)$ be the number of $f(j), j = 1,2, \ldots, n$,
such that $(f(j), n) = 1$. Show that $\psi(n)$ is multiplicative and that
$\psi(p^t) = p^{t-1} \psi(p)$. Conclude that $\psi(n) = n \prod_{p|n} \psi(p)/p$.

Suppose $a, b$ are relatively prime, and $1 \leq j \leq ab$. Then
$f(j)$ is relatively prime to $ab$ if and only if it is relatively
prime to both $a$ and $b$. Clearly, $f(j)$ is relatively prime to $a$
iff $f(j) \mod a$ is relatively prime to $a$, but since $f$ is
polynomial, $f(j) \mod a = f(j \mod a)$. Since the same holds for $b$,
$f(j)$ is relatively prime to $ab$ if and only if $f(j \mod a)$ is
relatively prime to $a$, and $f(j \mod b)$ is relatively prime to
$b$. Since $f(0) = f(a) \mod a$, there are exactly $\psi(a)\psi(b)$
numbers $1 \leq j \leq ab$ such that $f(j)$ is relatively prime to
$ab$ -- by the reasoning above, each such $j$ gives us a pair $(j \mod
a, j \mod b)$ (where $\mod$ are taken to be in $1, \ldots a$ and $1,
\ldots, b$ instead of $0, \ldots, a-1$ and $0, \ldots, b-1$) such that
$(f(j \mod a), a) = 1, (f(j \mod b), b) = 1$, and each pair $(r, s), 1
\leq r \leq a, 1 \leq s \leq b, (f(r), a) = 1, (f(s), b) = 1$ gives us
by Chinese remainder theorem a number $j$ in $1, \ldots, ab$ such that
$(f(j), ab) = 1$.

Now let $n = p^t$. For $1 \leq j \leq p^t$ we have $(f(j), p^t) =
(f(j), p) = (f(j) \mod p, p) = (f(j \mod p), p)$. Thus $(f(j), p^t) =
1$ iff $(f(j \mod p), p) = 1$. For $1 \leq j' \leq p$ there are
exactly $p^{t-1}$ different $j$, $1 \leq j \leq p^t$ such that $j \mod
p = j'$. Thus, $\psi(p^t) = p^{t-1} \psi(p)$.

Now, since $\psi$ is multiplicative, for $n = \prod_i p_i^{a_i}$ we
have $\psi(n) = \psi(\prod_i p_i^{a_i})= \prod_i \psi(p_i^{a_i}) =
\prod_i p^{a_i - 1} \psi(p) = \prod_i p^{a_i} \psi(p_i)/p_i = n
\prod_i \psi(p_i)/p_i$.

\paragraph{Ex. 2.24}
Supply the details to the proof of Theorem 3.

Since the Theorem 3 doesn't have any details left to supply, I assume
that the author means Theorem 4 here:

Theorem 4. $\sum q^{-\deg p(x)}$ diverges, where the sum is over all monic
irreducibles $p(x)\in k[x]$.

In the text authors show that $\sum q^{-\deg f(x)}$ diverges, and
$\sum q^{-2\deg f(x)}$ converges, where the sum is taken over all
monic polynomials. We now need to show that the same holds when one
takes the sum over only irreducibles.

Let $p_1, p_2, \ldots, p_{l(n})$ be all monic irreducible with degree
no larger than $n$, and define:

\begin{equation}
  \lambda(n) = \prod_{i=1}^{l(n)} \left(1-q^{-\deg(p_i)}\right)^{-1}
\end{equation}

Letting $l(n) = l$, we have:
\begin{eqnarray}
  \lambda(n) &=& \prod_{i=1}^{l} \sum_{j = 0}^\infty q^{-a_j
    \deg(p_i)} = \prod_{i=1}^{l} \sum_{j = 0}^\infty
  q^{-\deg(p_i^{a_j})} \\ &=& \sum \left(q^{\deg(p_1^{a_1})}
  q^{\deg(p_2^{a_2})} \ldots
  q^{\deg\left(p_{l}^{a_{l}}\right)}\right)^{-1} \\ &=& \sum
  \left(q^{\deg(p_1^{a_1})+\deg(p_2^{a_2})+ \ldots
    +\deg\left(p_{l}^{a_{l}}\right)}\right)^{-1} \\ &=& \sum
  \left(q^{\deg\left(p_1^{a_1}p_2^{a_2}\cdots p_{l}^{a_{l}}\right)}\right)^{-1}
\end{eqnarray}
where the sum is taken over all $l$-tuples of nonnegative integers
$(a_1, \ldots, a_l)$. Clearly, every polynomial $f \in k[x]$ with
$\deg(f) \leq n$ can be written as $f = p_1^{a_1}p_2^{a_2}\cdots
p_{l}^{a_{l}}$ for some $a_1, \ldots, a_l$. Therefore $\lambda(n) >
\sum_{\deg(f) \leq n} q^{-\deg(f)}$, and the latter sum diverges as $n
\to \infty$, and so $\lambda(n) \to \infty$. Therefore there are
infinitely many irreducible polynomials in $k[x]$.

Consider $\log \lambda(n)$. We have:

\begin{eqnarray}
  \log\lambda(n) &=&= -\sum_{i=1}^{l}
  \log\left(1-q^{-\deg(p_i)}\right) = \sum_{i=1}^{l}
  \sum_{m=1}^\infty (m q^{m\deg(p_i)})^{-1} \\ &=& q^{-\deg(p_1)} +
  q^{-\deg(p_2)} + \ldots + q^{-\deg(p_l)} + \sum_{i=1}^{l}
  \sum_{m=2}^\infty (m q^{m\deg(p_i)})^{-1}
\end{eqnarray}

Now, $\sum_{m=2}^\infty (m q^{m\deg(p_i)})^{-1} < \sum_{m=2}^\infty
q^{-m\deg(p_i)} = q^{-2\deg(p_i)}(1-q^{-\deg(p_i)})^{-1} \leq
2q^{-2\deg(p_i)}$. Therefore:

\begin{eqnarray}
  \log\lambda(n) &=& q^{-\deg(p_1)} + q^{-\deg(p_2)} + \ldots +
  q^{-\deg(p_l)} + \sum_{i=1}^{l} \sum_{m=2}^\infty (m
  q^{m\deg(p_i)})^{-1} \\ &\leq&\sum_{i=0}^\infty q^{-\deg(p_i)} + \sum_{i=1}^\infty 2 q^{-2\deg(p_i)}
\end{eqnarray}

We know that $\sum_{\deg(f) \leq n} q^{-2\deg(f)}$ converges, where
the sum is taken over all monic $f$, so $\sum_{i=1}^\infty 2
q^{-2\deg(p_i)}$ also converges. Therefore, if $\sum_{i=0}^\infty
q^{-\deg(p_i)}$, $\log \lambda(n)$ would have been bounded, which is
not true, since $\lambda(n)$ diverges.

\paragraph{Ex. 2.25}
Consider the function $\zeta(s) = \sum_{n=1}^\infty 1/n^s$. $\zeta$
is called the Riemann zeta function. It converges for s > 1. Prove the
formal identity (Euler's identity) $\zeta(s) = \prod_p (1-1/p^s)^{-1}$.

We have:

\begin{equation}
  \prod_p (1-1/p^s)^{-1} = \prod_p \sum_{i=0}^\infty \frac{1}{(p^i)^s}
\end{equation}

For $n = \prod p_i^{a_i}$, $n^s = \prod (p_i^{a_i})^s$, so $1/n^s =
1/\prod (p_i^{a_i})^s$ is a summand of the product above.

\paragraph{Ex. 2.26}
Verify the formal identities:
\begin{enumerate}[a)]
\item $\zeta(s)^{-1} = \sum \mu(n)/n^s$
  
We have:
\begin{equation}
  \left(\sum \frac{\mu(n)}{n^s}\right) \cdot \zeta(s)  = \sum_n \sum_{d|n} \frac{\mu(d)}{d^s(n/d)^s} = \sum_n \frac{1}{n^s}\sum_{d|n} \mu(d) = 1
\end{equation}
since $\sum_{d|n} \mu(d) = 0$ for $n \ne 1$.

\item $\zeta(s)^{2} = \sum \nu(n)/n^s$

We have:
\begin{equation}
  \zeta(s)^2 = \sum_n \sum_{d|n} \frac{1}{d^s(n/d)^s} = \sum_n
  \frac{1}{n^s}\sum_{d|n} 1 =\sum_n \frac{\nu(n)}{n^s}
\end{equation}

\item $\zeta(s)\zeta(s-1) = \sum \sigma(n)/n^s$
  
  We have:
\begin{eqnarray}
  \zeta(s)\zeta(s-1) &=& \left(\sum_{n=1}^\infty 1/n^s\right)
  \left(\sum_{n=1}^\infty 1/n^{s-1}\right) \\ &=&
  \left(\sum_{n=1}^\infty 1/n^s\right)\left(\sum_{n=1}^\infty
  n/n^s\right) \\
  &=& \sum_n \sum_{d|n} \frac{d}{(n/d)^sd^s} = \sum_n
  \frac{1}{n^s} \sum_{d|n} d = \sum_n \frac{\sigma(n)}{n^s}
\end{eqnarray}
\end{enumerate}

\paragraph{Ex. 2.27}
Show that $\sum 1/n$, the sum being over square free integers,
diverges. Conclude that $\prod_{p < N} (1+1/p) \to \infty$ as $N \to
\infty$. Since $e^x > 1 + x$, conclude that $\sum_{p < N} 1/p \to
\infty$.

Any nonnegative $n$ can be written uniquely as $n = a b^2$, $a$ square free. Thus, we have:
\begin{equation}
  \sum_{n = 1}^\infty \frac{1}{n} = \sum_{a \, \textrm{square free}}
  \sum_{b = 1}^\infty \frac{1}{ab^2} = \sum_{a \, \textrm{square
      free}} \frac{1}{a} \sum_{b = 1}^\infty \frac{1}{b^2} = \frac{\pi^2}{6}\sum_{a \, \textrm{square free}} \frac{1}{a}
\end{equation}

Thus, if $\sum_{a \, \textrm{square free}} 1/a$ converged, so would
$\sum_n 1/n$, which is known to diverge.

Therefore, we have:
\begin{equation}
  \prod_{p < N}(1+\frac{1}{p}) = \sum_a \frac{1}{a}
\end{equation}
where sum is taken over all square free $a$ such that their prime
factos are smaller than $N$. Thus $\prod_{p < N}(1+\frac{1}{p}) \to
\sum_{a \, \textrm{square free}} 1/a = \infty$. But since $e^x > 1 + x$,

\begin{equation}
  \prod_{p < N}(1+\frac{1}{p}) \leq \prod_{p<N} e^{1/p} = e^{\sum_{p<N} 1/p}
\end{equation}

If $\sum_p 1/p$ converged, $\prod_p (1+\frac{1}{p})$ would
have been bounded by $e^{\sum_{p} 1/p}$, but we have just shown it
diverges.

\section{Chapter 3}

\paragraph{Ex. 3.1}
Show that there are infinitely many primes congruent to $-1$ modulo $6$.

All primes are congruent either to $1$ or $-1$ modulo $6$. Suppose
$p_1, \ldots, p_n$ is a list of all primes congruent to $-1$ modulo
$6$. Consider a number $6p_1 p_2 \cdots p_n - 1$. It is relatively
prime to all of $p_i$, so all of its prime factors must be congruent
to $1$ modulo $6$. But if it was the case, it would also have been
congruent to $1$ modulo $6$, but we see that it is congruent to $-1$
modulo 6.

\paragraph{Ex. 3.2}
Construct addition and multiplication tables for $\Z/5\Z, \Z/8\Z, and
\Z/10\Z$.

Skipped.

\paragraph{Ex. 3.3}
Let $abc$ be the decimal representation for an integer between $1$ and
$1000$. Show that $abc$ is divisible by $3$ iff $a + b + c$ is
divisible by $3$. Show that the same result is true if we replace $3$
by $9$. Show that $abc$ is divisible by $11$ iff $a - b + c$ is
divisible by $11$.  Generalize to any number written in decimal
notation.

Let $n = \sum_{i=0}^k a_i 10^i$ for $0 \leq a_i \leq 9$. Then $n$ is
divisible by $3$ iff $n \equiv 0 \pmod 3$. But $10 \equiv 1 \mod 3$,
so $n \equiv \sum_{i=0}^k a_i \pmod 3$. For divisibility by $9$ and
$11$ note that $10 \equiv 1 \pmod 9$ and $10 \equiv -1 \pmod {11}$.

\paragraph{Ex. 3.4}
Show that the equation $3x^2 + 2 = y^2$ has no solution in integers.

We have either $y^2 \equiv 0 \pmod 3$ or $y^2 \equiv 1 \pmod 3$, while
the left hand side is congruent to $2$ modulo $3$.

\paragraph{Ex. 3.5}
Show that the equation $7x^2 + 2 = y^3$ has no solution in integers.

The left hand side is congruent to $2$ mod $7$, while possible values
of $y^3 \pmod 7$ are $0^3 = 0$, $1^3 = 1$, $2^3 = 8 \equiv 1 \pmod 7$,
$3^3 = 27 \equiv 6 \pmod 7$, $4^3 = 16 \cdot 4 \equiv 2 \cdot 4 \equiv
1 \pmod 7$, $5^3 = 25 \cdot 5 \equiv 4 \cdot 5 \equiv 20 \equiv 6
\pmod 7$, and $6^3 = 36 \cdot 6 \equiv 1 \cdot 6 \equiv 6 \pmod 7$.

\paragraph{Ex. 3.6}
Let an integer $n > 0$ be given. A set of integers $a_1, \ldots,
a_{\phi(n)}$ is called a reduced residue system modulo $n$ if they are
pairwise incongruent modulo $n$ and $(a_i, n) = 1$ for all $i$. If
$(a, n) = 1$, prove that $aa_1 , aa_2, \ldots, aa_{\phi(n)}$ is again
a reduced residue system modulo $n$.


\end{document}
